\section{Resultat}
Genom att genomföra de reflerktioner som tidigare beskrivits i avsnittet om metod har lett till framtagandet av dels mina 'best practices' men det står även klart vilken typ av teamledare jag har varit om Lewin's Leadership Style används som grund för bedömningen. Den stil som passar in är främst den delegerande på grund av att många arbetsuppgifter har tilldelats andra gruppmedlemmar med bättre kvalifikationer samt på grund av det faktum att gruppen har arbetat relativt autonomt på så sätt att varje medlem har fattat sina egna beslut. När det har funnits större beslut som ska fattas har det varit hela gruppen som fattat beslut och inte teamledaren vilket innebär att ledarskapsstilen även varit deltagande eller demokratisk. För att nämna den sista stilen, den auktoritära, har denna stil inte närvarat alls mycket på grund av att gruppen varit effektiv i att fatta beslut och det har på så sätt inte behövts någon ledare av den typen.
\newline \newline
Anledningen till att ledarskapsstilen behandlas här är att vilken typ av ledare man är till stor del påverkar vad som kommer att bli ledarens 'best practices'. Den första 'practicen' som hittades upptäcktes tidigt i projektet var att arbetet underlättades mycket av delegering av arbetsuppgifter. Detta framkom på grund av att delegeringen i princip var icke existerande till en början vilket ledde till att projektet var ostrukturerat och att gruppens medlemmar inte visste vad som behövde göras eller vad andra i gruppen arbetade med. Dessutom leddde bristen av struktur till bristande kundkontakt vilket innebar att projektetuppgiften inte var helt klar till en början. Denna 'practice' löste dessa problem vilket gör den till en av de viktigaste under projektet och det syns tydligt hur detta även spelar sin roll i arbetslivet. 
\newline \newline
Att hålla ett projekt strukturerat är en av grunderna till ett lyckat projekt men detta bygger även på att alla i gruppen får information om vad andra arbetar med, hur långt de kommit och hur det går för denne. Detta fås genom att hela tiden kommunicera med gruppen via någon kanal. I ett programmeringsprojekt blir detta väldigt tydligt när flera personer arbetar tillsammans med en uppgift. Problemet uppstår i detta fall när någon av dessa arbetar på egen hand vilket gör att den andra inte vet vad som skett eller vad som blivit fel när det finns brister i kommunikationen. Att bibehålla löpande kommunikation är med andra ord den andra 'practicen' som tillhör mina 'best practices'.

\begin{itemize}
	\item Delegering
	\item Kommunikation
	\item
\end{itemize}