\section{Resultat}
Genom att genomföra de reflerktioner som tidigare beskrivits i avsnittet om metod har lett till framtagandet av dels mina 'best practices' men det står även klart vilken typ av teamledare jag har varit om Lewin's Leadership Style används som grund för bedömningen. Den stil som passar in är främst den delegerande på grund av att många arbetsuppgifter har tilldelats andra gruppmedlemmar med bättre kvalifikationer samt på grund av det faktum att gruppen har arbetat relativt autonomt på så sätt att varje medlem har fattat sina egna beslut. När det har funnits större beslut som ska fattas har det varit hela gruppen som fattat beslut och inte teamledaren vilket innebär att ledarskapsstilen även varit deltagande eller demokratisk. För att nämna den sista stilen, den auktoritära, har denna stil inte närvarat alls mycket på grund av att gruppen varit effektiv i att fatta beslut och det har på så sätt inte behövts någon ledare av den typen.
\newline \newline
Anledningen till att ledarskapsstilen behandlas här är att vilken typ av ledare man är till stor del påverkar vad som kommer att bli ledarens 'best practices'. Den första 'practicen' som hittades upptäcktes tidigt i projektet var att arbetet underlättades mycket av delegering av arbetsuppgifter. Detta framkom på grund av att delegeringen i princip var icke existerande till en början vilket ledde till att projektet var ostrukturerat och att gruppens medlemmar inte visste vad som behövde göras eller vad andra i gruppen arbetade med. Dessutom leddde bristen av struktur till bristande kundkontakt vilket innebar att projektetuppgiften inte var helt klar till en början. Denna 'practice' löste dessa problem vilket gör den till en av de viktigaste under projektet och det syns tydligt hur detta även spelar sin roll i arbetslivet. 
\newline \newline
Att hålla ett projekt strukturerat är en av grunderna till ett lyckat projekt men detta bygger även på att alla i gruppen får information om vad andra arbetar med, hur långt de kommit och hur det går. Detta fås genom att hela tiden kommunicera med gruppen via någon kanal. I ett programmeringsprojekt blir detta väldigt tydligt när flera personer arbetar tillsammans med en uppgift vilket varit fallet även i detta projekt. Problemet uppstår i detta fall när någon av dessa arbetar på egen hand vilket gör att den andra inte vet vad som skett eller vad som blivit fel när det finns brister i kommunikationen. Ett exempel i detta projekt är att flera personer arbetat med samma problem men kommunikationen om vad som har gjorts av en person har i vissa fall inte nått de andra vilket skapat mycket förvirring och lett till att mycket onödig tid har spenderats på att försöka identifiera eventuella problem som skapats. Att bibehålla löpande kommunikation är med andra ord den andra 'practicen' som tillhör mina 'best practices'.
\newline \newline
Något som ligger nära att kommunicera väl med gruppen är att inte vara rädd att göra fel eller vara rädd att be om råd eller be om hjläp från gruppen. Det som gör att detta är viktigt när man arbetar med andra människor är att, oavsett om det är goda vänner eller helt okända människor man arbetar med, det kan underlätta för hela gruppen på flera olika sätt. Är någon gruppmedlem till exempel väldigt rädd att begå ett misstag är risken stor att denne istället blir ineffektiv i sitt arbete vilket kan bli en belastning för gruppen. Det är därför att föredra att en person vågar arbeta vidare och begå några misstag. När det gället att våga be om hjälp är detta viktigt då detta på samma sätt som att vara rädd för att begå misstag kan leda till spänningar inom gruppen vilket inte leder gruppen framåt i sitt arbete. En teamledare har ansvaret att försöka undvika detta genom att finnas som ett stöd för de gruppmedlemmar som kan befinna sig i en osäker situation och hänvisa till den information eller den person som kan hjälpa till. Detta har inte varit ett problem i projektet men i och med att gruppmedlemmar har vågat begå misstag och fråga om hjälp har detta framkommit vilket lett till att det anses vara en 'best practice'.
\newline \newline
De 'best practices' som identifierats under projektet sammanfattas av listan nedan:
\begin{itemize}
	\item Delegera arbete för att hålla projekt strukturerade
	\item Hålla en löpande kommunikation genom ett helt projekt
	\item Våga göra fel och be om hjälp
\end{itemize}