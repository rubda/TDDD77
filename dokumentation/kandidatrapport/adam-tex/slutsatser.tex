\section{Slutsatser}
Den nödvändiga informationen för att besvara de frågeställingar som tidigare ställts har genom rapporten samlats in. Den första frågan angående vad en 'best practice' är och hur den tillämpas besvarades redan i teorin där det sades att en 'best practice' är det kommersiella eller professionella tillvägagångssätt vilka är accepterade och föreskrivna som korrekta och mest effektiva. Genom rapporten har även insikten att en 'best practice' är specifik för varje ledare, projekt och projektgrupp framkommit. Det är även svårt att säga exakt hur en 'best practice' tillämpas på grund av att den är så specifik och dess tillämpning kan variera beroende på vilken miljö den används i. 
\newline \newline 
Den andra frågan som ställdes i frågeställningen krävde framtagandet av egna 'best practices' vilket gjordes genom att reflektera över genomförandet av teamledarrollen. Att delegering av arbete, vikten av att kommunicera både med kunden och inom gruppen samt att inte vara rädd att begå fel eller be om hjälp är ingen överraskning utan är snarare vad man kan förvänta sig i ett projekt som detta vilket leder till slutsatsen att de därför är 'best practices' för en teamledare i ett projekt som detta.
\newline \newline
Frågeställningen angående om det finns några generella 'best practices' besvarades genom en jämförelse mellan de som togs fram i rapporten och en representativ samling 'practices'. Det framkom att vissa element i olika personers 'best practices' kan vara liknande och därför göra dem tillämpbara för båda och därigenom olika typer av projekt. Dock framkom det att majoriteten av de mer vedertagna 'best practices' är designade för en teamledare med en roll som liknar mer en chef än en gruppmedlem vilket är fallet i detta projekt.Därför måste slutsatsen att det inte går att hitta några generella 'best practices' som kan anses vara det bästa tillvägagångssättet för alla grupper och projekt. Istället kan man säga att något som är en 'best practice' för en teamledare kan vara bra eller smarta practices. Denna slutsats utesluter dock inte att det kan finnas 'best practices' som gäller för några grupper vilket till störst del grundas i att mina 'practices' och de som jämförts med endast delar vissa element med varandra. Trots detta faktum finns möjligheten att det finns 'best practices' som kan fungera för flera olika projekt men att det är mycket viktigt att teamledarrollen ligger på samma nivå för att detta ska vara möjligt.