\section{Slutsatser}
I början av denna rapport ställdes följande frågor:
\begin{enumerate}
		\item Vad är och hur tillämpas 'best practices'?
		\item Vad har blivit mina 'best practices'?
		\item Finns det generella 'best practices' som kan tillämpas på alla projekt?
\end{enumerate}
Den nödvändiga informationen för att besvara dessa har genom rapporten samlats in. Den första frågan angående vad en 'best practice' är och hur den tillämpas besvarades redan i teorin där det sades att en 'best practice' är det kommersiella eller professionella tillvägagångssätt vilka är accepterade och föreskrivna som korrekta och mest effektiva. Genom rapporten har även insikten att en 'best practice' är specifik för varje ledare, projekt och projektgrupp framkommit. Det är även svårt att säga exakt hur en 'best practice' tillämpas på grund av att den är så specifik och dess tillämpning kan variera beroende på vilken miljö den används i. 
\newline \newline 
Den andra frågan som ställdes i frågeställningen krävde framtagandet av egna 'best practices' vilket gjordes genom att reflektera över genomförandet av teamledarrollen. Att delegering av arbete, vikten av att kommunicera både med kunden och inom gruppen samt att inte vara rädd att begå fel eller be om hjälp är ingen överraskning utan är snarare vad man kan förvänta sig i ett projekt som detta vilket leder till slutsatsen att de därför är 'best practices' för en teamledare i ett projekt som detta.
\newline \newline
I och med att det under hela projektet framkommit att varje grupp och varje projekt är unikt på något sätt måste slutsatsen att det inte går att hitta några generella 'best practices' som kan anses vara det bästa tillvägagångssättet. Istället kan man säga att något som är en 'best practice' för en teamledare kan vara bra eller smarta practices.