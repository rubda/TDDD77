\section{Diskussion}
I detta stycke diskuteras resultatet av rapporten samt den metod som användes för att komma fram till resultaten. Syftet med diskussionen är att bedömma om resultaten är pålitliga och om metoden fungerat väl i sammanhanget samt om det hade funnits någon annan metod som varit relevant för att komma fram till resultaten.

\subsection{Resultat}
De 'best practices' som fåtts från reflektionen bygger till stor del på misstag eller saker som borde utförts på ett annat sätt vilket innebär att, trots det faktum att reflektioner kan vara mycket personliga, bör anses vara pålitliga. Detta betyder dock inte att andra personer nödvändigtvis kommer komma fram till samma resultat då 'best practices' dels är personliga i det avseende att de påverkas av vilken typ av ledare man är men även att de är mer eller mindre skräddarsydda till detta projekt och denna projektgrupp. 

\subsection{Metod}
Den metod som huvudsakligen har använts i denna rapport har varit att utifrån egna reflektioner kring ageranden utforma 'best practices'. Dessutom har externa källor använts för att hitta kopplingar till andras 'best practices' för att på så sätt värdera de egna. 
\newline \newline
På grund av att metoden bygger på egna reflektioner uppstår en svårighet i att bibehålla ett objektivt synsätt. I denna rapport har jag därför försökt att vara så kritisk som mäjligt för att till största möjliga del eliminera detta problem men problematiken återstår. 
\newline \newline 
Ett alternativ till den valda och använda metoden hade istället kunnat vara att intervjua dels projektgruppen för att på så sätt få veta mer hur teamledarrollen har uppfylls från andra människors perspektiv men det hade även varit intressant att intervju en eller flera människor i arbetslivet som besitter rollen som teamledare eller liknande. Denna intervju hade då varit fokuserad på hur dessa människor fattar beslut och om de har någon egen 'best practice'.