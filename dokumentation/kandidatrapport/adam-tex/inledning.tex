\section{Inledning}
TODO
I dagens samhälle finns det en mängd olika typer av grupper och nästan alla har någon form av ledare. Det behöver inte röra sig om en formellt utsedd ledare utan kan även vara en informell ledare som på ett eller annat sätt har hamnat i den rollen. Att en grupp med människor har en ledare är inget nytt fenomen utan är något som återkommer genom hela våran historia och dessa ledare finns i en mängd olika former. Tittar man bara på dagens samhällen finns det allt ifrån monarker till folkvalda ledare i världen och alla har olika ledarstilar och ledaregenskaper. 
\newline \newline
Ledare genom historian, Om syfte, ledarskap och best practice

\subsection{Syfte}
När man själv befinner sig i en roll som ledare, eller närmare bestämt teamledare i detta fall, uppstår det en hel del frågor kring hur väl man genomför sitt uppdrag gentemot sin grupp och sig själv. På grund av att dessa frågeställningar uppstår finns det här en möjlighet att utforska området ytterligare med syftet att ta reda på vad jag i egenskap av teamledare har utfört bra och mindre bra...
\newline \newline
Reflektera över hur jag har utfört min roll som teamledare ur ett best practice perspektiv och de beslut som tagits står i relation till detta. Vad har varit mina 'best practices'?

\subsection{Frågeställning}
I listan nedan finns de frågeställningar som rapporten bygger på och som genom rapporten ska besvaras.
	\begin{enumerate}
		\item Vad är och hur tillämpas 'best practices'?
		\item Vad har blivit mina best practices?
		\item Är mina best practices kopplade till några befinntliga som andra har kommit fram till?
	\end{enumerate}

\subsection{Avgränsningar}
Mina ageranden, personlig åsikt om ageranden osv. \newline
När man undersöker en roll som teamledarrollen eller andra liknande ledarroller finns det väldigt många olika aspekter och områden att undersöka. I denna rapport läggs fokus på att hitta och utvärdera mina egna 'best practices' samt att försöka definiera vad min roll i projektet faktiskt har inneburit i praktiken vilket är en medveten avgränsning för att bibehålla en rimlig storlek på rapporten.
\newline \newline
Några av de områden som valts bort är ...
