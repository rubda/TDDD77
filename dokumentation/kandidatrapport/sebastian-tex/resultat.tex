\section{Resultat}
För att svara på de ursprungliga frågorna som ställdes i början av rapporten

\begin{enumerate}
	\item Vad är viktigt för att kunna skapa en arkitektur som är vidareutvecklingsbar?
	\item Hur ska arkitekturen se ut beroende på vad som ska uppnås utifrån kravspecifikationen?
	\item Hur och när bör man dela upp ett system i en arkitektur?
	\item Finns det några strategier som är bättre än andra för att skapa en arkitektur?
\end{enumerate}Så kan jag utifrån en analys av det som tagits upp i rapporten, komma fram till följande svar

\begin{enumerate}
	\item Ta inspiration och erfarenhet från arkitekturer som andra erfarna arkitekter har skrivit. Försök att med hjälp av de existerande arkitektoniska mönsterna göra en kombination som satisfierar just ditt problem. Var kreativ och jobba iterativt, gå tillbaka och gör ändringar i arkitekturen om så skulle behövas.
	\item Kravspecifikationen och kravdokumentet är båda grunderna till vad arkitekturen ska uppfylla. Sålänge arkitekturen inte bryter mot något i dessa så har man fria tyglar att utveckla arkitekturen på bästa möjliga sätt.
	\item Beroende på vad det är för system och framförallt hur stort det är så ska det delas upp om det är möjligt och om det förenklar förståelsen för systemet. Ett system som är väldigt komplext vill man gärna bryta ner i mindre delar som blir lättare att arbeta med var för sig. Detta medför även att flera utvecklare kan jobba samtidigt och det blir då mer kostnadseffektivt.
	\item Det finns ingen direkt formel för hur man skapar en bra arkitektur specifikt, utan det beror helt på vad det är för system som man ska göra. Det som ändå är bra att göra är att utgå ifrån de arkitektoniska mönster som finns.
\end{enumerate}Eftersom gruppen till en början inte hade särskilt stor koll på exakt hur systemet skulle göras och vilka delar som skulle behövas, eller någon direkt insikt på vilka arkitektoniska mönster som fanns eller hur man kunde använda dem. Så gjordes istället arkitekturen från grunden utan hjälp av några direkta arkitektoniska mönster, det som istället gjordes var att jag tog hjälp och tips av redan existerande arkitekturer för att bygga upp QuadOpts arkitektur på bästa sätt.
