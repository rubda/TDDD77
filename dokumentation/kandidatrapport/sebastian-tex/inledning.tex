\section{Inledning}
I denna rapport har jag tänkt att gå igenom vad som är viktigt att tänka på för att man ska kunna skapa en stabil arkitektur som är lätt att förstå och vidareutveckla, samt se för- och nackdelar med olika sätt som man kan bygga den på.
\newline
\newline
Som underlag till detta tänker jag använda källor så som tidsskrifter, internet och böcker.   

\subsection{Syfte}
Denna rapport är ett delmoment i kursen Kandidatprojekt i programvaruutveckling. Då min roll i projektgruppen är arkitekt, så har jag valt att gå in lite djupare på vad det viktiga är för att kunna skapa en bra arkitektur. Detta passar extra bra eftersom jag då kan ta med erfarenheter från projektet i min rapport.
Det jag vill åstadkomma med denna rapport är att väga fördelar mot nackdelar med olika sätt att göra arkitekturen på, samt skapa en bild av vad som är viktigt att tänka på vid utvecklingen av den.

\subsection{Frågeställning}
\begin{enumerate}
	\item Vad är viktigt för att kunna skapa en arkitektur som är vidareutvecklingsbar?
	\item Hur ska arkitekturen se ut beroende på vad som ska uppnås utifrån kravspecifikationen?
	\item Hur och när bör man dela upp ett system i en arkitektur?
	\item Finns det några strategier som är bättre än andra för att skapa en arkitektur?
\end{enumerate}

\subsection{Avgränsningar}
Eftersom det finns så mycket att skriva om hur man gör arkitekturer och det i sig skulle kunna bli en hel kurs, så har jag i denna rapport valt att fokusera mest på arbetet i det här kandidatprojektet.



