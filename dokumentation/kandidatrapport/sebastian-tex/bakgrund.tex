\section{Bakgrund}
I dagens allt större och komplexare dataprogram som består av allt fler komponenter som ska hänga ihop med varandra på olika vis och som ofta skrivs av allt fler utvecklare, så måste den följa en viss struktur för att det inte ska bli ett stort kaos av obegriplig kod och för att modulerna ska fungera korrekt med varandra. Till exempel användande av kodstandarder och designmönster för att de som utvecklar och även nya utvecklare lättare ska kunna sätta sig in och förstå sig på den.
\newline
\newline
En annan viktig del är även att dagens mjukvara ofta fortsätts att utvecklas lång tid efter första versionen med kontinuerligt nya uppdatering. Detta kräver att koden från början har en bra struktur och är ordentligt genomtänkt för sin uppgift och för framtida förändringar, så att inte allt för mycket behöver skrivas om då något nytt ska läggas till.
\newline
\newline
Arkitekturen kan ses som skeletet man bygger koden utifrån. Den ska fungera som en beskrivning för hur varje modul hänger ihop med resten av systemet.

\subsection{Prediktionsreglering}
Syftet med projektet var att implementera en optimeringsalgoritm för lösande av kvadratiska optimeringsproblem. Som är grunden till själva prediktionsregleringen som används för låta stridsflygplanet beräkna pilotens framtida manövrar, så det i ett tidigare skede kan optimera alla roder för att få ut maximal prestanda.
\newline
\newline
Arkitekturen beskrev just hur implementationen av optimeringsalgoritmen skulle fungera, vilka komponenter som behövdes och hur de skulle hänga ihop med varandra för att få ett fungerande system.

