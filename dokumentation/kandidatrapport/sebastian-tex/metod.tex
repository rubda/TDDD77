\section{Metod}
%Informationen kommer från en kombination av det som står på nätet och i böcker med mina egna erfarenheter från tidigare större programmeringsprojekt.
Uppdraget i kandidatprojektet var att göra en optimeringsalgoritm för prediktionsreglering, denna algoritm skulle anropas via en funktion i matlab eller via det egenutvecklade grafiska gränssnittet som via en parser som genererade koden som lösaren behövde för att lösa optimeringsproblemet. Det var dessa delar som arkitekturen behövde beskriva hur de skulle hänga ihop med varandra.
\newline
\newline
Den största delen av informationen i denna rapport kommer från kurslitteratur och artiklar som beskriver vad arkitekturen är och hur den bör utformas. En del kommer även från mina egna erfarenheter som arkitekt om vad som är viktigt i en arkitektur.
\newline
\newline
Arkitekturen till prediktionsregleringen har fått omarbetas flera gånger inom den blev bra och uppfyllde de krav som vi hade i kravspecifikationen. Detta på grund av att gruppen i början inte var säkra på hur allt skulle se ut och exakt vilka delar som skulle behövas. Det grafiska gränssnittet ingick t.ex. inte i början i arkitekturen, utan kom in senare i projektet som ett extra krav för att projektets timmar skulle kunna fyllas ut på ett bättre sätt och för att lösaren skulle kunna användas även utan hjälp från Matlab.

 
