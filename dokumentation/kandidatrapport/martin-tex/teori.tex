\section{Teori}

\subsection{Allmän design av biblioteket}
Alla operationer på matriser och vektorer bygger i grunden på matris structen som är definierad i matLib.h.
\lstinputlisting[language=C]{martin-tex/matrix_struct.c}
När en matris skapas så används följande funktion:
\lstinputlisting[language=C]{martin-tex/create_matrix.c}
Som skapar en $row \ x \ column$ stor matris. \\ \\
För att sätt in ett värde i matris används följande funktion:
\lstinputlisting[language=C]{martin-tex/insert_value.c}
Matriserna lagras row major så datan ligger lagrad enligt figur 1. 
\begin{figure}[h]
\center
\scalebox{0.8}{% Graphic for TeX using PGF
% Title: /home/martin/repositories/TDDD77/dokumentation/kandidatrapport/martin-tex/Diagram1.dia
% Creator: Dia v0.97.2
% CreationDate: Thu Apr 16 11:23:03 2015
% For: martin
% \usepackage{tikz}
% The following commands are not supported in PSTricks at present
% We define them conditionally, so when they are implemented,
% this pgf file will use them.
\ifx\du\undefined
  \newlength{\du}
\fi
\setlength{\du}{15\unitlength}
\begin{tikzpicture}
\pgftransformxscale{1.000000}
\pgftransformyscale{-1.000000}
\definecolor{dialinecolor}{rgb}{0.000000, 0.000000, 0.000000}
\pgfsetstrokecolor{dialinecolor}
\definecolor{dialinecolor}{rgb}{1.000000, 1.000000, 1.000000}
\pgfsetfillcolor{dialinecolor}
\pgfsetmiterjoin
\pgfsetdash{}{0pt}
\definecolor{dialinecolor}{rgb}{1.000000, 1.000000, 1.000000}
\pgfsetfillcolor{dialinecolor}
\fill (0.044745\du,1.818031\du)--(0.044745\du,16.818031\du)--(16.044745\du,16.818031\du)--(16.044745\du,1.818031\du)--cycle;
\pgfsetlinewidth{0.100000\du}
\definecolor{dialinecolor}{rgb}{0.500000, 0.500000, 0.500000}
\pgfsetstrokecolor{dialinecolor}
\draw (0.044745\du,5.568031\du)--(16.044745\du,5.568031\du);
\definecolor{dialinecolor}{rgb}{0.500000, 0.500000, 0.500000}
\pgfsetstrokecolor{dialinecolor}
\draw (0.044745\du,9.318031\du)--(16.044745\du,9.318031\du);
\definecolor{dialinecolor}{rgb}{0.500000, 0.500000, 0.500000}
\pgfsetstrokecolor{dialinecolor}
\draw (0.044745\du,13.068031\du)--(16.044745\du,13.068031\du);
\definecolor{dialinecolor}{rgb}{0.500000, 0.500000, 0.500000}
\pgfsetstrokecolor{dialinecolor}
\draw (4.044745\du,1.818031\du)--(4.044745\du,16.818031\du);
\definecolor{dialinecolor}{rgb}{0.500000, 0.500000, 0.500000}
\pgfsetstrokecolor{dialinecolor}
\draw (8.044745\du,1.818031\du)--(8.044745\du,16.818031\du);
\definecolor{dialinecolor}{rgb}{0.500000, 0.500000, 0.500000}
\pgfsetstrokecolor{dialinecolor}
\draw (12.044745\du,1.818031\du)--(12.044745\du,16.818031\du);
\pgfsetlinewidth{0.100000\du}
\definecolor{dialinecolor}{rgb}{0.000000, 0.000000, 0.000000}
\pgfsetstrokecolor{dialinecolor}
\draw (0.044745\du,1.818031\du)--(0.044745\du,16.818031\du)--(16.044745\du,16.818031\du)--(16.044745\du,1.818031\du)--cycle;
% setfont left to latex
\definecolor{dialinecolor}{rgb}{0.000000, 0.000000, 0.000000}
\pgfsetstrokecolor{dialinecolor}
\node[anchor=west] at (8.044745\du,9.318031\du){};
% setfont left to latex
\definecolor{dialinecolor}{rgb}{0.000000, 0.000000, 0.000000}
\pgfsetstrokecolor{dialinecolor}
\node[anchor=west] at (4.000000\du,8.000000\du){};
% setfont left to latex
\definecolor{dialinecolor}{rgb}{0.000000, 0.000000, 0.000000}
\pgfsetstrokecolor{dialinecolor}
\node[anchor=west] at (8.044745\du,9.318031\du){};
% setfont left to latex
\definecolor{dialinecolor}{rgb}{0.000000, 0.000000, 0.000000}
\pgfsetstrokecolor{dialinecolor}
\node[anchor=west] at (4.900000\du,6.700000\du){};
% setfont left to latex
\definecolor{dialinecolor}{rgb}{0.000000, 0.000000, 0.000000}
\pgfsetstrokecolor{dialinecolor}
\node[anchor=west] at (2.044745\du,3.693031\du){1};
% setfont left to latex
\definecolor{dialinecolor}{rgb}{0.000000, 0.000000, 0.000000}
\pgfsetstrokecolor{dialinecolor}
\node[anchor=west] at (6.044745\du,3.693031\du){2};
% setfont left to latex
\definecolor{dialinecolor}{rgb}{0.000000, 0.000000, 0.000000}
\pgfsetstrokecolor{dialinecolor}
\node[anchor=west] at (10.044745\du,3.693031\du){3};
% setfont left to latex
\definecolor{dialinecolor}{rgb}{0.000000, 0.000000, 0.000000}
\pgfsetstrokecolor{dialinecolor}
\node[anchor=west] at (14.044745\du,3.693031\du){4};
% setfont left to latex
\definecolor{dialinecolor}{rgb}{0.000000, 0.000000, 0.000000}
\pgfsetstrokecolor{dialinecolor}
\node[anchor=west] at (2.044745\du,7.443031\du){5};
% setfont left to latex
\definecolor{dialinecolor}{rgb}{0.000000, 0.000000, 0.000000}
\pgfsetstrokecolor{dialinecolor}
\node[anchor=west] at (6.044745\du,7.443031\du){6};
% setfont left to latex
\definecolor{dialinecolor}{rgb}{0.000000, 0.000000, 0.000000}
\pgfsetstrokecolor{dialinecolor}
\node[anchor=west] at (10.044745\du,7.443031\du){7};
% setfont left to latex
\definecolor{dialinecolor}{rgb}{0.000000, 0.000000, 0.000000}
\pgfsetstrokecolor{dialinecolor}
\node[anchor=west] at (14.044745\du,7.443031\du){8};
% setfont left to latex
\definecolor{dialinecolor}{rgb}{0.000000, 0.000000, 0.000000}
\pgfsetstrokecolor{dialinecolor}
\node[anchor=west] at (2.044745\du,11.193031\du){9};
% setfont left to latex
\definecolor{dialinecolor}{rgb}{0.000000, 0.000000, 0.000000}
\pgfsetstrokecolor{dialinecolor}
\node[anchor=west] at (6.044745\du,11.193031\du){10};
% setfont left to latex
\definecolor{dialinecolor}{rgb}{0.000000, 0.000000, 0.000000}
\pgfsetstrokecolor{dialinecolor}
\node[anchor=west] at (10.044745\du,11.193031\du){11};
% setfont left to latex
\definecolor{dialinecolor}{rgb}{0.000000, 0.000000, 0.000000}
\pgfsetstrokecolor{dialinecolor}
\node[anchor=west] at (14.044745\du,11.193031\du){12};
% setfont left to latex
\definecolor{dialinecolor}{rgb}{0.000000, 0.000000, 0.000000}
\pgfsetstrokecolor{dialinecolor}
\node[anchor=west] at (2.044745\du,14.943031\du){13};
% setfont left to latex
\definecolor{dialinecolor}{rgb}{0.000000, 0.000000, 0.000000}
\pgfsetstrokecolor{dialinecolor}
\node[anchor=west] at (6.044745\du,14.943031\du){14};
% setfont left to latex
\definecolor{dialinecolor}{rgb}{0.000000, 0.000000, 0.000000}
\pgfsetstrokecolor{dialinecolor}
\node[anchor=west] at (10.044745\du,14.943031\du){15};
% setfont left to latex
\definecolor{dialinecolor}{rgb}{0.000000, 0.000000, 0.000000}
\pgfsetstrokecolor{dialinecolor}
\node[anchor=west] at (14.044745\du,14.943031\du){16};
\end{tikzpicture}
}
\caption{Ordning som data i matrisen lagras på}
\end{figure}


\subsection{Tidskomplexitet på nuvarande implementationer operationer}
Alla tidskomplexiteter beräknas på nxn matriser.
Addition: $\mathcal{O}(n^2)$\\
Subtraktion: $\mathcal{O}(n^2)$\\
Multiplikation: $\mathcal{O}(n^3)$\\
Invers (crout och sedan lösa n ekvationssystem): $\mathcal{O}(n^3)$\\ 

\subsection{Algoritmiska förbättringar}
Multiplikation:\\
Istället för den naive algoritmen kan strassen algoritmen implementeras vilket reducerar tidskomplexiteten från $\mathcal{O}(n^3)$ till $\mathcal{O}(n^{2.807})$ (källa). \\
Inverse:\\
Istället för crout så kan inversen beräknas med strassen algoritmen och på så sätt så sänks tidskomplexiteten från $\mathcal{O}(n^3)$ till $\mathcal{O}(n^{2.807})$.

\subsection{Strukturella förbättringar}
När ett linjärt ekvationssystem på formen $Ax=b $ löses så bryts A ned till två matriser L och U där L är en undre triangulär matris och U är en övre triangulär matris. I matris structen skulle dessa matriser kunna sparas tillsammans med en boolean som säger om matrisen har modifierats så skulle man kunna undvika att beräkna U och L flera gånger för samma matris. Samma sätt skulle kunna användas för inversen till matriser. 