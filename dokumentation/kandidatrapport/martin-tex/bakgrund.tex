\section{Bakgrund}
Under kandidatprojektet Prediktionsreglering så skulle en lösare av kvadratiska komplexa problem tas fram. För detta behövdes ett matrisbibliotek väljas men inget bibliotek uppfyllde kraven. Dessa var:
\begin{enumerate}

\item Lättanvänt api
\item Bra prestanda
\item Platformsoberoende
\item Lätt att kompilera
\item Tar upp lite minne
\item Bra dokumenterad kod så man själv kan implementera förbättringar

\end{enumerate} 

Inget bibliotek uppfyllde alla dessa krav. De som undersöktes var:
\begin{itemize}

\item GNU Scientific library
\item LAPACK
\item ATLAS
\item NAG

\end{itemize}
Då togs beslutet att ta fram ett eget bibliotek som döptes till matLib. De matrisoperationer som behövdes var:
\begin{itemize}

\item addition
\item subtraktion
\item multiplikation
\item beräkna determinat
\item beräkna invers
\item lösa linjära ekvationssystem
\item gausselimination
\item transponering
\item skalärmultiplikation
\item radoperationer
\item kolumnoperationer

\end{itemize}
Efter implementation ska nu eventuella optimeringar undersökas. 

