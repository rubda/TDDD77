\section{Bakgrund}    
Detta projekt uppkom genom en industridoktorand vid Linköpings universitet som även arbetar åt Saab. Saab är en försvars- och säkerhetskoncern som är aktiv världen över. Saab  tillhandahåller produkter, tjänster och lösningar för både militära och civila ändamål. \citep{saabbrief}
\newline
\newline
En av de Saabs främsta och mest kända produkter är stridsflygplanet JAS 39 Gripen. Dock så kommer inte detta plan vara för evigt utan de håller redan på att utveckla nästa generations flygplan, det är i denna generations plan som MPC kan tänkas användas. För att kunna använda MPC i ett reglersystem så behövs optimeringsproblem lösas i realtid. Det är för denna uppgift lösaren utvecklas för att användas i simuleringar av systemet. 
\newline
\newline
Anledningarna till att Saab inte använder en kommersiell produkt är att de dels vill ha källkoden för lösaren, dels att de behöver använda systemet offline (de flesta kommersiella produkterna kräver att användaren är uppkopplad till internet).  
\newline
\newline
Saab vill även att lösaren ska kunna kallas från Matlab. 
