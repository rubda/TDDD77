\section{Diskussion}
Nedan följer analys av hur projektet har gått och arbetet ur ett vidare perspektiv.  

\subsection{Resultat}

\subsection{Metod}

\subsection{Arbetet i ett vidare sammanhang}

\subsubsection{Etiska och samhälleliga aspekter}
Att utveckla teknik åt ett företag som förser regeringar, myndigheter och företag med militära tjänster och produkter reser många etiska frågor. Hur kan man förhindra att känslig information hamnar i fel händer? Hur kan man veta att användaren har goda avsikter och inte använder vapnen för att förtrycka och förgöra? På Saabs hemsida kan man läsa att de har polisyn noll tolerans mot korruption och att det finns många åtgärder för att uppfylla detta. Nedanför i figur 1 visas en grafisk överblick över deras huvudsakliga åtgärder.
\\
\leavevmode
\begin{figure}[h]
	\centering
	\includegraphics[scale=1.5]{grafik/modell_zero_corruption_1140x640.png}
	\caption{Zero tolerance}
	\label{fig:zero tolerance}	
\end{figure}  
\\
Till exempel gör Saab alltid riskanalyser i samband med affärer för avgöra om det finns risk för korruption. De undersöker risker med vart affären äger rum, vem köparen är, hur upphandlingen går till och hur köparen kom i kontakt med företaget. Om riskerna inte gick att eliminera eller inte var hanterbara drar sig Saab ur affären.             
\\
Det finns alltid en risk med att sälja vapen. I en perfekt värld existerar det inga vapen, vi lever dock inte i en perfekt värld.


\subsubsection{Miljöaspekter}
Flygplan inte speciellt miljövänliga!