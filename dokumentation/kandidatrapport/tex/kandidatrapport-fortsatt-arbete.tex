\section{Fortsatt arbete}
Det här projektet erbjuder många olika möjligheter till fortsatt arbete för alla inblandade parter. Detta inkluderar alla olika delar av projektet. Det fortsatta arbetet innebär till största delen att vidareutveckla dessa delar för att förbättra befintliga funktioner och eventuellt lägga till nya funktioner till det redan befintliga programmet. 
\newline \newline
Till att börja med kan det ligga i Saabs och Daniel Simons intresse att fortsätta arbeta med produkten, framför allt lösaren. Det mest relevanta för denna part skulle till exempel vara att integrera produkten med Saabs system, ARES, vilket ligger på en hög nivå i det avseendet att produkten måste vara mycket pålitlig och effektiv för att detta ska bli aktuellt. Därför går det även att se hur denna part skulle kunna anse att det är en bra idé att vidareutveckla den levererade produkten för att uppfylla de standarder och krav som finns. För denna part finns det även aspekten att arbeta vidare med produkten i ett rent akademiskt syfte för simuleringar relaterade till Simulink och MATLAB. Detta är möjligt då Saab och Daniel Simon kommer ha tillgång till all källkod som krävs för dessa typer av arbete.
\newline \newline
Vidare går det även att titta på vidare arbete ur kandidatgruppens perspektiv men den potential som finns är begränsad på grund av att produkten utvecklats åt SAAB. Trots detta finns möjligheten att arbeta vidare på produkten. Det som framför allt skulle vara önskvärt att arbeta vidare med är att genomföra omfattande tester av lösaren med fler typer av testdata vilket varit problematiskt på grund av storleken på datan. För kandidatgruppen skulle fortsatt arbete även kunna innebära att den nuvarande lösaren skrivs om för att tillämpa en annan optimeringsmetod än den nuvarande. Den som i nuläget används är Active set men det är även möjligt att implementera till exempel Interior point metoden.
\newline \newline
Undersöks istället enbart produkten utan att ta hänsyn till någon av parternas perspektiv är det lättare att se möjligheter för vidareutveckling. Den del av produkten som är kopplad till MATLAB har väldigt begränsad möjlighet till vidareutveckling i och med att denna del i stort sett är en översättning av lösaren vilket gör att det istället är mer relevant att titta på lösaren vilken har stor potential att arbetas vidare med. I lösaren finns det både optimeringsmöjligheter samt möjligheten att lägga till eller ändra funktioner. När det gäller optimering kan man tänka sig att vidare arbete innebär att programmet optimeras för att vara effektivare med de tillgängliga resurserna för att på så sätt bli snabbare. De funktioner som redan finns implementerad som kan arbetas vidare med kan till exempel vara hur lösaren hittar sin startpunkt för optimeringen.
\newline \newline
De andra två delarna av programmet, det grafiska gränssnittet och parsern, har också potential att arbetas vidare med. Det är dock viktigt att poängtera att fortsatt arbete med dessa delar inte är direkt relaterade till hur lösaren fungerar eller prestandan hos denna vilket kan ses som att det är mindre värdefullt att arbeta vidare med detta jämfört med lösaren som trots allt är programmets huvudsakliga komponent.
\newline \newline
En brist som funnits under projektet har varit viss avsaknad av testfall vilket kommer av att testfallen är stora och omständiga att ta fram. Skulle det finnas tillgång till fler testfall är ytterligare tester ett område som kan arbetas vidare på. Detta kan till exempel leda till att fler områden i lösaren kan optimeras för att bli snabbare eller att eventuella brister uppdagas. Brister skulle till exempel kunna vara ovanliga specialfall som i nuläget inte har beaktats och som endast i enstaka fall kan leda till något problem.