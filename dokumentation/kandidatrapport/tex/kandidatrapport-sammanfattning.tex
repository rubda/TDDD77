\section{Sammanfattning}
Denna kandidatrapport studerar en projektuppgift som har utförts av en grupp studenter på Linköpings universitet. Uppgiften har givits av en industridoktorand på Saab och härstammar från reglering av styrsystem i stridsflygplanet JAS 39 gripen. Det som har utförts av kandidatgruppen är en  kvadratisk optimeringslösare som även kan köras från MATLAB. Kandidatgruppen har studerat olika optimeringsalgoritmer och valt en att implementera utifrån vilken kunden rekommenderade och vilken som verkade lättast. Undersökningar har gjorts om det går att implementera optimeringsalgoritmen i programspråket C, om lösaren kan bli lika snabb som den kommersiella produkten Gurobi och om projektet går att utföra utan någon speciell utvecklingsmetodik. I resulten går det att se att det gick att implementera optimeringsalgoritmen, att lösaren inte kunde bli lika snabb som Gurobi och att kandidatgruppen inte använde någon speciell utvecklingsmetodik. Slutsatser kandidatgruppen har dragit är att valet av optimeringsalgoritm inte var helt genomtänkt, att mer mer tid och resurser hade lösaren kanske kunnat blivit lika snabb som Gurobi och att arbetet fungerade tillfredsställande utan någon speciell utvecklingsmetod.     