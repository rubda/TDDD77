\section{Slutsatser}
De slutsatser kandidatgruppen har kommit fram till är att valet av metod till optimeringsalgoritmen inte var helt genomtänkt. Anledningen till detta är att kandidatgruppen saknade förkunskaper inom området och bör ha lagt fler timmar på utbildning inom optimeringsproblem i detta område vilket hade lett till ökad förståelse för området på ett generellt sätt vilket hade lett till att de algoritmer som varit kandidater hade kunnat bedömas på ett mer kvalificerat sätt för att kunna göra ett mer kvalificerat val. 
\newline
\newline
Kandidatgruppen ångrar inte valet av att ta med kravet att QuadOpt ska vara jämbördig med Gurobi vilket inte uppnåts. Genom att misslyckats med detta har kandidatgruppen lärt sig förhandla krav med en kund och göra sitt yttersta för att matcha en kommersiell produkt vilket inte anses vara en enkel uppgift med tanke på den begränsade tiden kandidatgruppen haft vilket ett företag som utvecklar en produkt som Gurobi troligtvis haft större tillgång av. Dock kan man dra slutsatsen att det är möjligt att skapa en likvärdig produkt rent prestandamässigt om det funnits mer tid dels för utveckling men framför allt, som tidigare nämnts, utbildning.
\newline
\newline
Kandidatgruppen ångrar inte heller valet av att inte gå in i iterationerna utan en konkret utvecklingsmetod eftersom arbetet fungerade på ett tillfredsställande sätt utan en sådan metod.  