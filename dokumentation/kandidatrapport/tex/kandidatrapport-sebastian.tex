\section{Sebastian Fast - Uppbyggnad av stabila arkitekturer}
\subsection{Inledning}
I denna rapport har jag tänkt att gå igenom vad som är viktigt att tänka på för att man ska kunna skapa en stabil arkitektur som är lätt att förstå och vidareutveckla, samt se för- och nackdelar med olika sätt som man kan bygga den på.
\\
Som underlag till detta tänker jag använda källor så som tidsskrifter, internet och böcker.   
\subsubsection{Syfte}
Denna rapport är ett delmoment i kursen Kandidatprojekt i programvaruutveckling. Då min roll i projektgruppen är arkitekt, så har jag valt att gå in lite djupare på vad det viktiga är för att kunna skapa en bra arkitektur. Detta passar extra bra eftersom jag då kan ta med erfarenheter från projektet i min rapport.
Det jag vill åstadkomma med denna rapport är att väga fördelar mot nackdelar med olika sätt att göra arkitekturen på, samt skapa en bild av vad som är viktigt att tänka på vid utvecklingen av den.
\subsubsection{Frågeställning}
\begin{enumerate}
	\item Vad är viktigt för att kunna skapa en arkitektur som är vidareutvecklingsbar?
	\item Hur ska arkitekturen se ut beroende på vad som ska uppnås utifrån kravspecifikationen?
\end{enumerate}
\subsubsection{Avgränsningar}
Eftersom det finns så mycket att skriva om hur man gör arkitekturer och det i sig skulle kunna bli en hel kurs, så har jag i denna rapport valt att fokusera mest på det här kandidatprojektet.
\subsection{Bakgrund}
I dagens allt större och komplexare datorprogram som ofta skrivs av allt fler utvecklare är det ett måste att den följer en viss struktur för att det inte ska bli ett kaos av obegriplig kod. Till exempel användande av kodstandarder och designmönster för att de som utvecklar och även nya utvecklare lättare ska kunna sätta sig in och förstå sig på den.
\\\\
En annan viktig del är även att dagens mjukvara ofta fortsätts att utvecklas lång tid efter första versionen med kontinuerligt nya uppdatering. Detta kräver att koden från början har en bra struktur och är ordentligt genomtänkt för sin uppgift och för framtida förändringar, så att inte allt för mycket behöver skrivas om då något nytt ska läggas till.
\subsection{Teori}
http://en.wikipedia.org/wiki/Software\_architecture
\subsection{Metod}
Informationen kommer från en kombination av det som står på nätet och i böcker med mina egna erfarenheter från tidigare större programmeringsprojekt.
\subsection{Resultat}
\subsection{Diskussion}
\subsubsection{Resultat}
\subsubsection{Metod}
Är den använda metoden pålitlig? Skulle det kunnat gjorts på något annat sätt för att få ett bättre resultat?
\subsection{Slutsatser}
