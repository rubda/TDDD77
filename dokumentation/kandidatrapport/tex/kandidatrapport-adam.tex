\section{Adam Sestorp - Projektledning}
\emph{Best practices utifrån egna erfarenheter, beslut och iakttagelser i egenskap av teamledare.}

\subsection{Inledning}
TODO
I dagens samhälle finns det ... oändligt antal ... typer av grupper och nästan alla har någon form av ledare. Det behöver inte röra sig om en formellt utsedd ledare utan kan även vara en informell ledare som på ett eller annat sätt har hamnat i den rollen. Att en grupp med människor har en ledare är inget nytt fenomen utan är något som återkommer genom hela våran historia. 
\newline \newline
Ledare genom historian, Om syfte, ledarskap och best practice

\subsubsection{Syfte}
När man själv befinner sig i en roll som ledare, eller närmare bestämt teamledare i detta fall, uppstår det en hel del frågor kring hur väl man genomför sitt uppdrag gentemot sin grupp och sig själv. På grund av att dessa frågeställningar uppstår finns det här en möjlighet att utforska området ytterligare med syftet att ta reda på vad jag i egenskap av teamledare har utfört bra och mindre bra. Det är dessutom önskvärt att relatera detta till vad som kallas 'best practices' och utifrån detta forma mina egna.
\newline \newline
Reflektera över hur jag har utfört min roll som teamledare ur ett best practice perspektiv och de beslut som tagits står i relation till detta. Vad har varit mina 'best practices'?

\subsubsection{Frågeställning}
I listan nedan finns de frågeställningar som rapporten bygger på och som genom rapporten ska besvaras.
	\begin{enumerate}
		\item Vad är och hur tillämpas 'best practices'?
		\item På vilken ledarskapsnivå har jag agerat? (enligt någon modell?)
		\item Vad har blivit mina best practices?
		\item Är mina best practices kopplade till några befinntliga som andra har kommit fram till?
	\end{enumerate}
	
\subsubsection{Avgränsningar}
Mina ageranden, personlig åsikt om ageranden osv. \newline
När man undersöker en roll som teamledarrollen eller andra liknande ledarroller finns det väldigt många olika aspekter och områden att undersöka. I denna rapport läggs fokus på att hitta och utvärdera mina egna 'best practices' samt att försöka definiera vad min roll i projektet faktiskt har inneburit i praktiken vilket är en medveten avgränsning för att bibehålla en rimlig storlek på rapporten.
\newline \newline
Några av de områden som valts bort är ...

\subsection{Bakgrund}
Jag har under kursen Kandidatprojekt i programvaruutveckling haft rollen som teamledare för en grupp bestående av sex andra studenter vid Linköpings universitet. Som teamledare har man i detta fall ansvar för en rad olika administrativa uppgifter vilket bland annat innebär ansvar för projektets framgång och uppfyllande av de krav som finns uppsatta. 
\newline \newline
Under denna process väcks många frågor kring hur väl den antagna rollen uppfylls jämfört med arbetslivet vilket lett till denna jämförelse mellan mina prestationer som teamledare och de 'best practises' som kan tillämpas.

\subsection{Teori}
Denna rapport har egentligen två huvudbegrepp, 'best practice' och teamledare. 'Best practices' är enligt Oxfords engelska uppslagsverk som "Commercial or professional procedures that are accepted or prescribed as being correct or most effective". Löst översatt till svenska betyder detta kommersiella eller professionella tillvägagångssätt vilka är accepterade och föreskrivna som korrekta och mest effektiva. Med andra ord är en 'best practice' det sätt en person på bäst sätt kan utföra en uppgift vilket i fallet som teamledare innebär att genomföra rollen på bästa sätt. 
\newline \newline
Teamledare och ledarskapsmodell etc...
"Commercial or professional procedures that are accepted or prescribed as being correct or most effective"
\newline \newline
Ref: http://www.oxforddictionaries.com/definition/english/best-practice 2015-04-16

\subsection{Metod}
Den metod som används för att ta fram relevant data till denna rapport kommer till stor del bestå i att reflektera över mina egna ageranden som teamledare för att ta fram 'best practices'. För att uppnå bästa möjliga resultat hämtas den största datamängden från iteration tre det vill säga den sista iterationen i projektet samt från tidigare iterationer. Dessutom kommer en jämförelse mellan mer vedertagna 'best practices' och de som tas fram här att genomföras. För att genomföra detta hämtas relevant data från källor så som böcker, artiklar och hemsidor. På grund av att 'best practices' i väldigt många fall är personliga åsikter och att det kan vara svårt att hitta en 'best practice' som är generell för detta område fås jämförelsen genom att jämföra flera olika 'best practices' för att ta fram en representativ modell. Detta kommer på grund av den personliga aspekten ske på ett väldigt kritiskt sätt för att få en så vetenskaplig jämförelse som möjligt.

\subsection{Resultat}
Resultat av mina reflektioner av beslut mm.

\subsection{Diskussion}
Vad kunde gjorts annorlunda/bättre utifrån resultaten?

\subsubsection{Resultat}
Pålitliga resultat? 

\subsubsection{Metod}
Hade undersökningen kunnat utföras på annat sätt?
\newline
Tex intervjuer

\subsection{Slutsatser}
Återkoppla till frågeställningen.
