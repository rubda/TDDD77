Datorsystem\section{Johan Isaksson}
	\subsection{Inledning}
	Då jag (Johan Isaksson) är testledare i gruppen faller det naturligt att skriva om något testrelaterat. Därför ska denna del av rapporten handla om hur kod och andra aspekter av ett program ska testas. Även vilka olika sätt man kan testa på samt en utvärdering av hur bra de fungerar, både i praktiken och på papper. Informationen i denna del kommer att komma från läroböcker, internet och från egna erfarenheter under projektet. 
	
	
	\subsubsection{Syfte}
	Syftet med denna del av rapporten är att klargöra hur bra olika metodiker inom mjukvarutestniong fungerar.
	
	
	\subsubsection{Frågeställning}
	
	\subsubsection{Avgränsningar}
	Metoder vi valt att inte använda
	
	
	\subsection{Bakgrund}
	beskriv testningen genom projektet
	
	
	\subsection{Teori}
	
	\subsection{Metod}
	Black Box
	Acceptans
	
	
	\subsection{Resultat}
	reslutat av testningen, hur mycket fel hittateds, hastighet och exakthet
	resultat av metod, användes den på korrekt sätt, vad uppfylldes inte
	
	
	\subsection{Diskussion}
	varför blev det som det blev i "Resultat"
	annan lämpligare approach?
	
	
	\subsubsection{Resultat}
	
	\subsubsection{Metod}
	\subsection{Slutsatser}
	Vad ska man tänka på tills nästa gång