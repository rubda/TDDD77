\section{Teori}
Syftet med projektet är att ta fram ett snabbt och lättutvecklat program som löser konvexa kvadratiska optimeringsproblem. Det vi har gjort är att ha jämfört olika algoritmer mot varandra i fråga om hastighet, robusthet samt implementerbarhet. Problemet med detta är att algoritmerna som vi har testat varierar i hastighet beroende på hur stora matriserna som ska lösas är samt hur de är strukturerade, även vilket programmringsspråk och plattform som algoritmen är implementerade på spelar roll.

\subsection{Liknande problem}
Det finns sedan tidigare redan många andra program som löser liknande optimeringsproblem. Anledning till att vi gör ett nytt är för att de som redan finns antingen är breda och riktar in sig på många olika problem, vilket gör dem långsammare eftersom de är sämre optimerade för just vårt problem. Ett annat är att de oftast inte är öppen källkod, vilket gör att man inte får med källkoden till programmet så att man själv inte kan vidareutveckla det. Ett annat problem med de som faktiskt har öppen källkod är att man inte säkert kan veta om de använder sig av tredjepartskod som har någon annan licens, eller om de utvecklare som varit med och skrivit koden skulle vilja ha ersättning för det de gjort.
Tanken med vårt program är att vi ska äga det, men Saab AB ska ha rätten att vidareutveckla och använda det kommersiellt.

\subsection{Algoritmer och metoder}
Vår huvudsakliga källa för de algoritmer och metoder vi använt kommer från boken Numerical Optimization av Jorge Nocedal och Stephen J. Wright. Denna bok var ett tips från vår kund eftersom den innehöll tre algoritmer som han tyckte skulle lämpa sig bäst för problemet. Det var sedan vårt jobb att välja den av dessa algoritmer som skulle vara snabbast, enklast att implementera och utveckla för vårt problem.
TODO: Skriv vilken vi valde och varför, jämför den mot de andra algoritmerna som fanns att välja mellan... Ändra källhänvisningen enligt Harvardmodellen...