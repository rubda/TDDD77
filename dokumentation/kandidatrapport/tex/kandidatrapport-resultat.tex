\section{Resultat}

De frågeställningar som ställdes i början av dokumentet var:
\begin{enumerate}
\item Går det att implementera en kvadratisk optimeringslösare i programspråket C?
\item Kan vi implementera ett system som löser kvadratiska optimeringsproblem snabbare än Gurobi?
\item Kan projektet utföras utan någon speciell utvecklingsmetodik? 
\end{enumerate}

Svaren på dessa frågor:
\begin{enumerate}
\item Ja, det går att implementera en kvadratisk optimeringslösare i programspråket C. Detta eftersom kandidatgruppen har lyckats med att göra det. Den kvadratiska optimeringslösaren som skrevs i programspråket C var active-set metoden. 

\item Nej, ett system som kan lösa kvadratiska optimeringsproblem snabbare än Gurobi kan inte utvecklas med den tid som har funnits till hands. Kandidatgruppens sju medlemmar hade 270 timmar var att fördela projektet på, varav många timmar lades ner på dokumentation och diverse seminarier. Utan dokumentation och seminarier skulle optimeringsalgoritmen kunnat blivit bättre.

\item Ja, projektet kan utföras utan någon speciell utvecklingsmetodik. Däremot så kan man jobba efter en viss utvecklingsmetodik utan att veta om det, dvs en skräddarsydd utvecklingsmetodik. Kandidatgruppen i detta fall gick in i iterationerna utan en utvecklingsmetodik, men under arbetetsgång skapades en sorts utvecklingsmetodik som hjälpte till att utveckla optimeringsalgoritmen.
\end{enumerate}
	
	
\subsection{Gruppens gemensamma erfarenheter}
En lärdom vi tagit under projektets gång är vikten av att hålla regelbundna möten med kunden i början av projektet för att klargöra vad som verkligen ska göras. Många frågor uppkommer och utan en klar kravspecifikation blir det svårt att planera projektet.
\newline
\newline
Vi har haft goda erfarenheter med att använda molntjänster som Github och Google drive för hantering av kod och dokumentation, så länge vi gör upp vilka som skriver vad för att undvika konflikter.  
\textcolor{red}{ utveckla}

\subsection{Översikt över de individuella utredningarna}