\section{Inledning}
Idag designas många flygplan så att de är instabila om de skulle flyga ut reglering. Detta är för att skapa ett plan som är mer lättrörligt. Även plan som inte är designade för att vara instabila behöver reglering för att förhindra att planet eventuellt blir instabilt. För att sköta reglering används diverse reglersystem. \citep{airplanestability}
\newline
\newline
Prediktionsreglering (eng. model predictive control, MPC) är en reglerleringsmetod som 
bygger på att man med hjälp av en modell av systemet antar hur systemet kommer reagera på styrningen och på så sätt reglera systemets framtida tillstånd beroende på nuvarande och tidigare tillstånd. Detta har stor industriell relevans.\citep[2]{ir}
\newline
\newline
Projektuppgiften från Saab är att välja ut och implementera en optimeringsalgoritm som löser detta optimeringsproblem. Då det redan finns kommersiella produkter som löser detta problem ska slutprodukten jämföras med en av dessa. Kunden gav förslaget Gurobi som är en av de mer välkända och etablerade optimeringsprogrammen.

\subsection{Syfte}
Ett av de syften som finns är att projektgruppens medlemmar systematiskt ska tillämpa de kunskaper som förvärvats under studietiden framför allt inom programmering och datalogi men även utvecklingsmetodik. Dessutom ska medlemmarna tillgodogöra sig innehållet i relevant facklitteratur samt relatera denna till sitt arbete. 
Det finns även ett syfte från ett annat perspektiv där projektgruppen ska skapa en produkt som skapar värde för kunden samt att projektgruppen ska få en inblick i arbetslivet och hur en utvecklingsprocess kan se ut där. Att lära känna nya människor och lära sig att samarbeta med dem är också en viktig del av projektet. \citep{tddd77}


\subsection{Frågeställning}
	\begin{enumerate}
		\item Går det att implementera en kvadratisk optimeringslösare i programspråket C med tiden som har getts kandidatgruppen?
		\item Kan kandidatgruppen implementera ett system som löser kvadratiska optimeringsproblem snabbare än Gurobi?
		\item Kan projektet utföras utan någon speciell utvecklingsmetodik? 
	\end{enumerate}

\subsection{Avgränsningar}
I ett projekt med samma omfattning som detta måste det finnas avgränsningar som begränsar projektet i olika avseenden. De avgränsningar eller begränsningar som finns i detta projekt rör främst tillgängliga resurser och den ämneskunskap som krävs.
\newline
\newline
I denna rapport behandlas endast kvadratiska konvexa optimeringsproblem där målfrunktionen är kvadratisk och bivillkoren är linjära. En avgränsning som var ett krav från kunden var att lösaren skulle vara implementerad i programmeringsspråket C så det görs inget aktivt val av språk. Parserns grafiska gränssnitt var också tvunget att efterlikna CVXGEN så det behövdes inte tas fram ett nytt gränssnitt. 
