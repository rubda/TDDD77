\section{Inledning}
Idag när samhället har nått en utveckligsnivå där datorer är en viktig del i den dagliga verksamheten för både företag och privatpersoner har datorer även fått en plats i många av de produkter företagen producerar. Dessa integrerade datorer kan ge användaren en hjälpande hand för att denne ska kunna utföra fler och mer komplicerade uppgifter. Detta innebär även att avancerade konstruktioner så som flygplan innehåller många olika system där datorer spelar en viktig roll. Ett sådant exempel är autopiloten som hjälper piloten att styra flygplanet. Hastigheter och positioner mäts och informationen skickas till datorerna som baserat på mätsignalerna reglerar gaspådrag och mototer så att planet flyger stabilt. Det finns många olika sorters reglerstrategier att använda och en av dem är prediktionsreglering. 
\newline
\newline
Prediktionsreglering (eng. model predictive control, MPC) är en avancerad reglerstrategi som fått stort genomslag i industrin och som forskningsområde under de senaste decennierna. Styrkan ligger i att den ger högpresterande styrsystem som kan köras utan något högre ingripande under längre tid. Prediktionsreglering är dock beräkningskrävande och en modell krävs. \citep[2]{ir}
\newline
\newline
Saab har under flera års tid forskat kring prediktionsregleringsalgoritmer för att kunna användas i nästa generations stridsflygsystem. Dessa algoritmer kräver att flygplanets styrdator löser avancerade optimeringsproblem i realtid. (Se bilaga A)
\newline
\newline
Projektuppgiften som fåtts av Saab är att välja ut och implementera en optimeringsalgoritm som löser detta optimeringsproblem. Då det redan finns kommersiella produkter som löser detta problem ska projektet jämföras med en av dessa. Saab gav förslaget Gurobi som är en av de mer välkända och etablerade optimeringsprogrammen. Vårat projekt ska lösa kvadratiska optimeringsproblem minst lika snabbt som Gurobi.     

\subsection{Syfte}
Ett av de syften som finns med att genomföra detta projekt är att projektgruppens medlemmar systematiskt ska tillämpa de kunskaper som förvärvats under studietiden framför allt inom programmering och datalogi men även projektutvecklingsmetodik. Dessutom ska medlemmarna tillgodogöra sig innehållet i relevant facklitteratur samt relatera denna till sitt arbete. 
Det finns även ett syfte från ett annat perspektiv där projektgruppen ska skapa en produkt som skapar värde för kunden samt att projektgruppen ska få en inblick i arbetslivet och hur en utvecklingsprocess kan se ut där. Lära känna nya människor och lära sig att sammarbeta med dem är också en viktig del av projektet. 

\subsection{Frågeställning}
	\begin{enumerate}
		\item Går det att implementera en kvadratisk optimeringslösare i programspråket C?
		\item Kan vi implementera ett system som löser kvadratiska optimeringsproblem snabbare än Gurobi?
		\item Kan projektet utföras utan någon speciell utvecklingsmetodik? 
	\end{enumerate}

\subsection{Avgränsningar}
I ett projekt med samma omfattning som detta måste det finnas avgränsningar som begränsar projektet i olika avseenden. De avgränsningar eller begränsningar som finns i detta projekt rör främst tillgängliga resurser och den ämneskunskap som krävs.
\\ \\
Ur ett resursperspektiv är det tid som är en avgränsande faktor. Projektetgruppen har totalt 2100 timmar till sitt förfogande där varje gruppmedlem förväntas arbete ungefär 300 timmar. Dessutom avgränsas framför allt projektets tidiga skede av den kunskap inom kvadratisk optimering vilket inte lärts ut under utbildningen och därför gett upphov till mycket utbildning inom detta.