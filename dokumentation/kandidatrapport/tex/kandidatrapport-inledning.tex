\section{Inledning}
Idag när samhället har nått en utveckligsnivå där datorer är en viktig del i den dagliga verksamheten för både företag och privatpersoner har datorer även fått en plats i många av de produkter företagen producerar. Dessa integrerade datorer kan ge användaren en hjälpande hand för att denne ska kunna utföra fler och mer komplicerade uppgifter. Detta innebär även att avancerade konstruktioner så som flygplan innehåller många olika system där datorer spelar en viktig roll. Ett sådant system är reglersystemet som ser till att flygplanet är stabilt oavsett inputen från. Det finns många olika sorters regleringsmetoder att använda och en av dem är prediktionsreglering. 
\newline
\newline
Prediktionsreglering (eng. model predictive control, MPC) är en reglerleringsmetod som 
bygger på att man med hjälp av en modell av systemet antar hur systemet kommer reagera på styrningen och på så sätt reglera systemets framtida tillstånd beroende på nuvarande och tidigare tillstånd. Detta har stor industriell relevans.\citep[2]{ir}
\newline
\newline
Saab har under flera års tid forskat kring prediktionsregleringsalgoritmer för att kunna användas i nästa generations stridsflygsystem. Dessa algoritmer kräver att flygplanets styrdator löser avancerade optimeringsproblem i realtid. (Se bilaga A)
\textcolor{red}{ använd label och ref}
\newline
\newline
Projektuppgiften som fåtts av Saab är att välja ut och implementera en optimeringsalgoritm som löser detta optimeringsproblem. Då det redan finns kommersiella produkter som löser detta problem ska slutprodukten jämföras med en av dessa. Kunden gav förslaget Gurobi som är en av de mer välkända och etablerade optimeringsprogrammen. Ett ursprungligt krav var att lösaren skulle kunna lösa att problem lika snabbt som Gurobi men detta visade sig senare vara svårt att uppnå så det kravet förhandlades bort. 

\subsection{Syfte}
Ett av de syften som finns med att genomföra detta projekt är att projektgruppens medlemmar systematiskt ska tillämpa de kunskaper som förvärvats under studietiden framför allt inom programmering och datalogi men även projektutvecklingsmetodik. Dessutom ska medlemmarna tillgodogöra sig innehållet i relevant facklitteratur samt relatera denna till sitt arbete. 
Det finns även ett syfte från ett annat perspektiv där projektgruppen ska skapa en produkt som skapar värde för kunden samt att projektgruppen ska få en inblick i arbetslivet och hur en utvecklingsprocess kan se ut där. Lära känna nya människor och lära sig att samarbeta med dem är också en viktig del av projektet. 

\subsection{Frågeställning}
	\begin{enumerate}
		\item Går det att implementera en kvadratisk optimeringslösare i programspråket C?
		\item Kan vi implementera ett system som löser kvadratiska optimeringsproblem snabbare än Gurobi?
		\item Kan projektet utföras utan någon speciell utvecklingsmetodik? 
	\end{enumerate}

\subsection{Avgränsningar}
I ett projekt med samma omfattning som detta måste det finnas avgränsningar som begränsar projektet i olika avseenden. De avgränsningar eller begränsningar som finns i detta projekt rör främst tillgängliga resurser och den ämneskunskap som krävs.
\\ \\
Ur ett resursperspektiv är det tid som är en avgränsande faktor. Projektetgruppen har totalt 2100 timmar till sitt förfogande där varje gruppmedlem förväntas arbete ungefär 300 timmar. Dessutom avgränsas framför allt projektets tidiga skede av den kunskap inom kvadratisk optimering vilket inte lärts ut under utbildningen och därför gett upphov till mycket utbildning inom detta.
\newline
\newline
En annan avgränsning som fanns under den inledande fasen av projektet var begränsning av utrymme att arbeta på. Vi fick ingen utrymme av universitetet utan fick boka rum för möten och arbetssessioner. Detta var problematiskt då man endast kan boka rum 24 timmar i förväg genom universitetet vilket ledde till att detta tog mycket tid. Både att boka och ta sig till rummen man fick då de ofta låg långt bort och att man ibland var tvungen att byta rum under dagen om någon annan hade boket det. Senare under projektet så fick vi tillgång till ett kontor som tillhör IDA vid universitetet med hjälp av Patrick Doherty, professor vid IDA. 