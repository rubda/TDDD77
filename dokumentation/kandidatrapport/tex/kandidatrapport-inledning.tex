\section{Inledning}
Idag när samhället har nått en utveckligsnivå där datorer är en viktig del i den dagliga verksamheten för både företag och privatpersoner har datorer även fått en plats i många av de produkter företagen producerar. Dessa integrerade datorer ger användaren en hjälpande hand för att denne ska kunna utföra fler och mer komplicerade uppgifter. Detta innebär även att avancerade konstruktioner så som flygplan innehåller många olika system där datorer spelar en viktig roll.

Detta projekt kommer från SAAB och stridflygplanet JAS 39 Gripen. I flygplanet används en prediktionsreglering för att förutse pilotens kommandon och skapa styrsignaler som styr flygplanets roder. Dessa signaler behöver även optimeras vilket ger upphov till ett kvadratiskt komplext problem som behöver lösas så snabbt som möjligt. Projektuppgiften som fåtts av SAAB är att välja ut och implementera en optimeringsalgoritm som löser detta optimeringsproblem.

\subsection{Syfte}
Ett av de syften som finns med att genomföra detta projekt är att projektgruppens medlemmar systematiskt ska tillämpa de kunskaper som förvärvats under studietiden framför allt inom programmering och datalogi men även projektutvecklingsmetodik. Dessutom ska medlemmarna tillgodogöra sig innehållet i relevant facklitteratur samt relatera denna till sitt arbete. 
Det finns även ett syfte från ett annat perspektiv där projektgruppen ska skapa en produkt som skapar värde för kunden samt att projektgruppen ska få en inblick i arbetslivet och hur en utvecklingsprocess kan se ut där.

\subsection{Frågeställning}
1. Vilka erfarenheter kan man få genom att genomföra ett programmeringsprojekt?
2. Vilka erfarenheter är överförbara till kommande projekt?
3. Har vi noterat någon förbättring av de lärdomar vi dragit från tidigare projekt?
4. Vilket stöd kan man få genom att använda och följa upp SEMAT Kernel ALPHA:s?
5. Specifika frågeställningar.



\subsection{Avgränsningar}
I ett projekt med samma omfattning som detta måste det finnas avgränsningar som begränsar projektet i olika avseenden. De avgränsningar eller begränsningar som finns i detta projekt rör främst tillgängliga resurser och den ämneskunskap som krävs.

Ur ett resursperspektiv är det tid som är en avgränsande faktor. Projektetgruppen har totalt 2100 timmar till sitt förfogande där varje gruppmedlem förväntas arbete ungefär 300 timmar. Dessutom avgränsas framför allt projektets tidiga skede av den kunskap inom kvadratisk optimering vilket inte lärts ut under utbildningen och därför gett upphov till mycket utbildning inom detta.