\section{Metod} \label{avsnitt:metod}
Då projektet redan har ett byggsystem implementerat i Make behövs ett likvärdigt byggsystem i SCons implementeras parallellt. På grund av begränsningarna nämnda i avsnitt \ref{avsnitt:avgransningar} kommer endast kompilering och länkning implementeras i SCons och det är endast på dessa områden prestandan för de två byggsystem kommer att mätas. Då det befintiliga Make-systemet innefattade extra funktionalitet för att köra till exempel enhetstester, skapades även ett nytt minimalistiskt Make-system baserat på det befintliga som enbart kompilerade och länkade kod. De nya byggsystemen implementeras på en separat gren i versionshanteringssystemet för att undvika eventuella konflikter.
\newline
\newline
Tiden det tar för byggsystemen att bygga all kod i projektet mäts med hjälp av UNIX-kommandot time och den slutgiltiga resultatet tas som medelvärdet av fem mätningar. Byggsystemens förmåga att bygga parallellt kommer att testas med körningar på en, två, fyra, åtta och 16 trådar.
\newline
\newline
All prestandatestning kommer att ske på konfigurationen i tabell \ref{tabell:konfig}.

\begin{table}[h!]
  \centering
  \begin{tabular}{|l|l|}
    \hline
    \textbf{Hårdvara} & \textbf{Version} \\ \hline
    MacBook Air & Early 2015 \\ \hline
    \textbf{Mjukvara} & \textbf{Version} \\ \hline
    OS X & Yosemite 10.10 \\ \hline
    GNU Make & 3.81 \\ \hline
    SCons & 2.3.4 \\ \hline
    gcc & 4.8.4 \\ \hline
  \end{tabular}
  \caption{Hård- och mjukvarukonfiguration för prestandatesterna.}
  \label{tabell:konfig}
\end{table}

För att utreda vilket byggsystem som är enklast att använda och vidareutveckla presenterades de båda systemen för kandidatgruppen som fick ge sina synpunkter på till exempel läsbarhet och förväntad underhållbarhet. Dessa synpunkter tillsammans med gruppens erfarenheter under projektets gång kan också ge svar på hur valet av byggsystem har påverkat arbetet.
