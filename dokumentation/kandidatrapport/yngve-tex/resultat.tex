\section{Resultat}
I avsnitt \ref{avsnitt:fragestallning} framfördes frågeställningarna:

\begin{enumerate}
\item Vilket av de två byggsystemen har bäst prestanda?
\item Vilket byggsystem är lättast att använda och utveckla?
\item Hur hade projektet påverkats av ett annat val av byggsystem?
\end{enumerate}

\noindent Efter undersökningen beskriven i \ref{avsnitt:metod} kan svaren på dessa ges här.

\subsection{Prestanda}
\begin{table}[h!]
  \centering
  \begin{tabular}{|l|l|l|}
    \hline
    \textbf{Trådar} & \textbf{Make} (s) & \textbf{SCons} (s) \\ \hline
    1 & 2.65 & 3.62 \\ \hline
    2 & 1.56 & 2.34 \\ \hline
    4 & 1.32 & 2.19 \\ \hline
    8 & 1.32 & 2.20 \\ \hline
    16 & 1.34 & 2.25 \\ \hline
  \end{tabular}
  \caption{Resultat från prestandatester.}
  \label{tabell:prestanda}
\end{table}

\todo{Graf}

\subsection{Användning och vidareutveckling}
\todo{Användning skiljer sig inget, bara olika kommandon, make är dock mer känt och behöver ingen vidare förklaring (ingen i gruppen visste om scons)} \\
\todo{SCons skulle varit lättare, då alla tyckte sconsfilerna var mer förståeliga och dessutom hade alla erfarenhet av python}

\subsection{Byggsystemets påverkan på projektet}
\todo{Hänvisa till att liten prestandaskillnad -> ingen större påverkan}
