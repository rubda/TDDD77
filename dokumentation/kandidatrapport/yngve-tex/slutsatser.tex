\section{Slutsatser}
Målet med denna rapport var att utvärdera två typer av byggsystem med projektet Prediktionsreglering som utgångspunkt. Det visade sig att valet av byggsystem inte spelade särksilt stor roll prestandamässigt för just det här projektet. Vad som däremot kunde ha påverkats var förmågan att utveckla byggsystemet i takt med projektets gång. Kandidatgruppen tyckte utan undantag att SCons verkade lättare att förstå och utveckla.

Om det enklare systemet SCons hade valts från början hade förmodligen hela kandidatgruppen kunnat vara mer delaktiga i att underhålla byggsystemet. Detta skulle antagligen lett till större kunskapsspridning och mindre beroende mot byggsystemsansvarige när något går fel. Det hade i sin tur frigjort mer tid för den byggsystemsansvarige att bidra till projektet på andra sätt.

En möjlig förbättring för projektet är således att implementera ett byggsystem i SCons för ökad produktivitet och mer kunskapsspridning. Nackdelen med detta är förstås att SCons inte alls har lika stor spridning som alternativet Make, som i det närmaste kan kallas industristandard.
