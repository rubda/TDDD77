\section{Resultat}	
	Följande del beskriver hur arbetet med efterforskningen gick samt hur testen utfördes.
	\subsection{Efterforskning}
	Det finns otroligt mycket information om mjukvarutestning men samtidigt är ämnet ganska vagt då testning beror så mycket på just vad som ska testas. I vårt projekt visade det sig att ''Black Box''-testning var den metod som överlägset lämpade sig bäst. ''Black Box'' går ut på att man sätter en ''svart låda'' över det som ska testas, så att man endast kan se in- och utdatan. Sedan kollar man ifall udatan är den förväntade. ''Black Box'' anses bra i detta projekt eftersom hela Quadopt är uppbyggd utav många små funktioner och resultateten som de ska ge tillbaka är oftast kända på förhand. Ett exempel på detta är matrisaritmetiken där resultatet, av till exempel en multiplikation, går att räkna ut ganska enkelt på papper. Enligt ''The Art of Software Testing'' ska dessa test utgå ifrån kravspecifikationen och andra dokument som beskriver vad produkten ska ha för funktionalitet. Boken beskriver också ''Black Box'' som en utmattande testteknik. Med det menar de att man borde testa alla möjliga indata till den svarta lådan och se så att svaren stämmer. Detta är precis som det står i boken i praktiken omöjligt. Speciellt i vårt projekt där det enda som begränsar antalet olika sätt en matris, bestående utav tal, kan se ut på är datorns minne. \newline
I projektet fanns dock ofta behovet av se en funktions lösningsgång, och då är ''Black Box'' en väldigt dålig metod. En metod som då lämpar sig bättre är ''White Box'' testning, som innebär att man kollar på den interna strukturen i en funktion. Därefter kan man kolla, efter vald indata, om lösningsgången är den väntade. I ''The Art of Software Testing'' står det att även denna metod kan problematisk då antalet lösningsgångar kan vara väldigt många. För att se om det ens är rimligt att utföra dessa test kan man kolla på funktionens cyklomatiska komplexitet. Cyklomatisk komplexitet innebära att man gör en graf över funktionen där de möjliga stadierna i lösningsgången är noder, och de möjliga lösningsvägarna är bågar. I ''Structured testing'' \citep{structest} står det att om denna komplexitet är för stor ökar antalet fel som programmeraren gjort väldigt fort, och samtidigt blir det i stort sett omöjligt att uföra några ''White Box''-test då fel kan uppstå på så många ställen. \\ 	
För att säkerställa projektets krav behövdes bara en testmetod till, och det var en metod för att mäta prestanda. Den som valdes var ''Load testing'' som innebär att man belastar programmet med mycket data och så kollar man på hur bra det fungerar. I projektets fall gavs lösaren många problem och kollade på hur fort det gick i förhållande till andra lösare. \newline	
Det skulle kunnat varit så att GUI:t hade haft högre prioritet än vad det hade. I det fallet hade olika typer av UX- användartest varit nödvändiga för att kvalitetsssäkra produkten. Men eftersom GUI:t beställdes utifrån kundens personliga behov, var det tydligt definierat redan från början att det var av låg prioritet. 
	
	\subsection{Praktik}
	Som beskrivet tidigare finns det i stort sett oändligt många kombinationer av in- och utdata.	För att då kunna utföra ''Black Box''-testerna behövdes antalet testfall reduceras. Detta åstakoms genom att ha möten med kunden som klargjorde att indata till programmet alltid skulle vara giltig. Det reducerade antal testfall enormt mycket, men som beskrivet i resultatet av efterforskningen finns det även väldigt mycket giltig indata. Exempelvis för matrisaritmetiken. Dessa test gick också att reducera genom att de flesta operationer är triviala och endast kräver numeriska test såsom nolldivision och flyttalsfel. Genom att även utnyttja ''White Box''-tekniken gick det att utforma olika ''Black Box''-test som tog olika vägar genom koden och på så sätt bara skapa ett test för varje väg. Denna teknik utnyttjades endast på mindre funktioner såsom moduler för att antalet olika fall skulle begränsas till något som var rimligt. \newline
	För lösaren gick det att applicera samma metod, eftersom dess funktionalitet bygger mycket på underliggande funktioner. Det som skilde sig gentemot småfunktionerna var att nu behandlades oftast rader eller kolumner i matriser istället för enskillda element. Detta ledde till att de flesta test kollade på kanterna utav det tillåtna området. Alltså kunde ett test vara att försöka hämta ut en radvektor på rad -1 ur en matris, vilket skulle vara ogiltigt.\newline
Vid granskning av testresultat från git, Travis (verktyg för Continuous integration) och gruppmedlemmar visade det sig att majoriteten fel av bestod utav två typer: ''Assertion''s som misslyckats och ''Segmentations fault''. En ''Assertion'' är ett test inuti koden som avbryter exekveringen om testet inte går bra. ''Seqmentation fault'' är ett programmeringsfel som resulterat i en ogiltig läsning eller skrivning till minnet. Felen som uppstod var väldigt utspridda och olika. Det som de flesta hade gemensamt var dock att de låg på en låg nivå, alltså i bottenfunktionerna. \newline	
	För att testa algoritmens hastighet stötte gruppen på oväntade problem, lösarna var för snabba.	Då varje testkörning tog 0.00 sekunder för alla lösarna förutom MATLAB behövdes testen köras många gånger för att se en tydlig skillnad. Anledningen till att MATLAB är långsammare är för att den är oerhört generell men förmodligen inte gjorts med fokus på att vara snabb. Gurobi är snabb eftersom dens enda uppgift är att lösa sådana här problem och har arbetats på under lång tid. Vår algoritm är snabb eftersom den inte är generell, alltså bryr vi oss inte om vissa specialfall som vi aldrig kommer stöta på. \newline
	För att då testa deras prestanda fick lösarna lösa olika optimeringsproblem många gånger, ofta upp emot 1000 gånger, för att kunna skilja deras egentliga hastighet. Att köra testen på vår lösare och i matlab var enkelt då vi hade funktionalitet för att konvertera matriser från MATLABs definition till vår, och vice versa. Däremot så var det krångligare i gurobi eftersom tiden som var allokerad för utbildning av programmet var begränsad. Det tvingade gruppen till att mata in prblemet på ekvationsform vilket gjorde att vi var tvungna att köra testen med få variabler. Till exempel så skulle testfallet med 92 variabler ta väldigt lång tid att konvertera och mata in. \newline
	
	
	\subsection{Enhetstester}
	Under projektet har många enhetstest skrivits (framförallt för matrisbiblioteket). Dessa test har skrivits innan och under kodningen och sedan utförts direkt efter att koden blivit klar. Det som var intressant med dessa test var mängden av fel som upptäcktes direkt. Det ledde till att utvecklingen blev mycket effektiv då fel kunde åtgärdas direkt.
	
	\subsection{Integrationstester}
	De modultester som planerades och utfördes under projektet var framförallt lösarens funktioner. De var uppbyggda utav sammansättningar av enheter från matrisbiblioteket och andra strukturer. Dessa test utfördes vanligtvis samtidigt som implementeringen pågick. Detta för att hela tiden se till att rätt protokoll och gränsnitt användes.\\
De systemtester som utförts är tester utav lösaren och sublösaren. Dessa ansågs vara tillräcklig komplexa för att ses som system. 
	
	
	\subsection{Acceptanstester}
	De acceptanstester som utförts under projektet är framförallt prestandatester. Detta för att det enda kravet lösaren hade var att den skulle vara ungefär lika snabb som den kommersiella lösaren gurobi. De delar som testats då var underfunktioner till lösaren, såsom matrisbiblioteket och subproblemslösaren.
	
	\subsection{Misstag}
	Det hände att vissa modul- och systemtester skedde innan de underliggande enheterna blivit testade. Ett exempel på det är lösaren som vi var ivriga att få igång och började testa tidigt. Då den inte fungerade korrekt gjorde detta att det tog lång tid att hitta felet som förmodligen hade upptäckts mycket snabbare om bara rätt testprocess hade använts.
	
	