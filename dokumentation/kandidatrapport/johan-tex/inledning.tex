\section{Inledning}
	Då jag (Johan Isaksson) är testledare i gruppen faller det naturligt att skriva om något testrelaterat. Därför ska denna del av rapporten handla om hur kod och andra aspekter av ett program ska testas. Även ett par olika sätt man kan testa på samt reflektioner över hur bra de fungerar i praktiken mot vad som står i böcker. Informationen som kommer att behandlas i denna del i denna del kommer att komma från läroböcker, internet och från egna erfarenheter under projektet. 
	
	
	\subsection{Syfte}
	Syftet med denna del av rapporten är att klargöra hur bra olika metodiker inom mjukvarutestning fungerar samt hur och när de ska användas. Även att visa hur lätt eller svår en metod kan vara i praktiken samt vilka andra aspekter det finns som påverkar resultatet av metoden.
	
	
	\subsection{Frågeställning}
	I denna rapport kommer följande frågor att besvaras:
	\begin{enumerate}
	\item{Hur, när och varför ska programmet testas?}
	\item{Gjorde vi på bästa sätt?}
	\item{Vad kan göras bättre?}
	\end{enumerate}
	
	\subsection{Avgränsningar}
	De delar som valts att tas bort i detta avsnitt av rapporten är de testmetoder som inte använts under projektets gång. Detta för att tiden är begränsad och för att få så mycket förståelse som möjligt för just det som tas med. Huvudsakligen kommer det att handla om områdena Enhetstester, Integrationtester (Modultester och Systemtester) och Acceptanstester. \newline
	Anledningen till att rapporten har denna inriktning är för att mjukvarutestning är ett väldigt brett ämne. Vissa delar måste skippas för att hålla rapporten till en rimlig storlek. Dessutom så känns det mest relevant att diskutera de aspekter som har utövats under projektet, något man har erfarenhet utav.