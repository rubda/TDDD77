\section{Diskussion}
	Denna sektion kommer att förklara varför resultatet blev som det blev och vad som kan förbättras.	
	
	\subsection{Resultat}
	Då detta är första gången som gruppen jobbar med testning som en aktiv del av ett projektet var planering och utförandet ganska svårt att ta sig in i. I början testades precis som planerat små enheter men på grund utav bristfällig utbildning påbörjades sytem- och modultesterna för tidigt. Gruppen ville tidigt testa lösaren då dens funktionalitet var av högsta prioritet, men eftersom subproblemslösaren var bristfälligt testad (ty den bristfälliga utbildningen) och i själva verket inte fungerade, även om den klarade de test som var skrivna, så var stora delar av koden tvungen att testas om. Detta hade lätt undvikts om vi hade lagt mer tidigt på utbildningen. Eftersom det var i slutet av iteration 2 som subproblemslösaren först blev färdigställd gjordes många test i onödan, men dock berodde inte det på något testrelaterat utan snarare på de beslut som togs. 
	
	\subsection{Metod}
	Tanken var att genom hela projektet använda oss utav en ''Bottom-up''-strategi. Men eftersom organisationen i gruppen angående vem som skulle göra vad var dålig i början glömdes större delar av testningen bort. Detta ledde till att vissa funktioner felaktigt antogs fungera och att en ''Top-down''-strategi användes senare vilket inte fungerade så bra då majoriteten av fel hittades på låg nivå i koden.
	Då projektet ändå var ganska entydigt, med att vi skulle implementera en algoritm som löser konvexa kvadratiska optimeringsproblem, så var det ett enkelt beslut att köra enhetstester och integrationstester. Det framgick tidigt att det var framförallt funktionalitet som behövde testas, och därför behövde planeringen av testningen endast göras grovt i början och sedan specificeras när testningen närmade sig. Det som skulle behövas göras tidigare är framförallt att organisera och schemalägga testning för att testare skall kunna veta vad de har för ansvar.