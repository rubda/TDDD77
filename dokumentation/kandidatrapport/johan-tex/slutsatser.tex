\section{Slutsatser}
	Det som jag har lärt mig utav detta projekt angående testning är vikten av att tidigit definiera projektets omfång och syfte. Utan en tydlig förståelse av projektet blir testplannering en väldigt svår uppgift och en mindre lyckad plannering påverkar arbetet genom hela projekttiden på ett negativt sätt. Dock krävs det även en tydlig kravspecifikation och en idé om hur arkitekturen ser ut för att kunna definiera målen med testningen. Detta för att sedan kunna plannera och skapa de testfall som ska säkerställa mjukvarans funktionalitet. \\
Det som skulle kunna förbättras är som nämnt tidigare organisationen. När gruppen arbetade med testningen enligt beskrivningar i testplanen flöt arbetet på bra, men så fort en enhet/modul/system ansågs vara klar behövde gruppen organiseras upp igen för fortsatt testarbete. Om mer tid allokerats till testplanering i början hade förmodligen testningen gått som på räls. \\
Valet av strategi med ''Black Box''-test var utan tvivel rätt då ingen annan metodik passade lika bra in med vårt project. Däremot så hade testen kunnat utformas snabbare och bättre. Det som gjorde att vissa test var mindre bra var bristen på utbildning. När en grupp med väldigt lite förkunskaper får en uppgift som denna måste mer tid allokeras till utbildning. Om mer tid hade lagts på utbildning hade testarbetet kanske inte kunnat börja ända från början, men betydligt tidigare.
	
\subsection{Sammanfattande svar på frågeställningar}
	\begin{enumerate}
	\item Programmet ska testas med hjälp utav ''Black Box''-test på de delar där komplexiteten inte är för stor och resultatet är känt på förhand. På de delar där insikt i kodens lösning krävs bör istället ''White Box''-test användas. ''Black Box''-test kan därefter tas fram med hjälp utav dessa. Testen bör utföras direkt efter implementationen och helst utav utvecklaren själv, eftersom denne oftast är mest insatt. Programmet ska testas på detta sätt om det mestadels är funktionellt uppbyggt. Om det istället är inriktat på användarbarhet eller liknande är denna strategi dålig.
	\item De flesta test utfördes på korrekt sätt och i rätt ordning. Men som beskrivet under ''Misstag'' så användes ibland fel strategi vid fel tidpunkt.
	\item För att förbättra resultatet utav testprocessen krävs en djup förståelse utav både uppgiften och den implementerade koden. Dessa faktorer skulle i detta projekt kunnat förbättras genom mer utbildning.
	\end{enumerate}
	
