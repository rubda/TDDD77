\section{Resultat}
Inledningsvis i detta dokument ställdes frågorna:

\begin{enumerate}
  \item Vad är rätt nivå gällande kvalité?
  \item Vad krävs för att en slutprodukt eller tjänst ska vara av tillräckligt god kvalité?
  \item Har slutprodukten i kandidatarbetet tillräckligt god kvalité?
\end{enumerate}

\noindent Efter att ha fördjupat i teorin inom kvalitetssäkring och utvärderat arbetet som har genomförts så är svaren:

\begin{enumerate}
  \item Rätt nivå gällande kvalité är det som har kommit överens om med kunden. Om kund och leverantör har godkänt en kravspecifikation mellan varandra, så räcker det för leverantören att uppfylla de krav som finns i en eventuell kravspecifikation för att uppnå rätt nivå gällande kvalité.
  \item Det som krävs för att uppnå en slutprodukt eller tjänst av tillräckligt god kvalité är att man har en gedigen utvecklingsprocess. Genom att följa Shewhart cykeln kan man få struktur och agera om något går fel.  
  \item Ja, slutprodukten håller tillräckligt god kvalité. Detta på grund av att kandidatgruppen uppfyllde alla de kraven som hade specificerats. Dock så bör det noteras att ett krav förhandlades bort.
\end{enumerate}