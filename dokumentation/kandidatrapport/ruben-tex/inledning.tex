\section{Inledning}
Kvalitetssäkring evaluerar och modifierar en organisations metoder för att se till att en slutprodukt håller en god nivå, om inte utmärkt nivå.
\newline
\newline
Att lyckas uppnå en god kvalitetsstandard genom alla steg vid utvecklandet av en produkt eller tjänst kräver struktur, en struktur som sätts upp av en enskild individ eller en grupp av personer med relevant kunskap. I denna individuella rapport tänker jag ta upp vad som krävs för att uppnå en sådan struktur som sedan kan hjälpa till att generera en produkt eller tjänst av tillräckligt hög kvalité till kundens behov.

\subsection{Syfte}
Syftet med denna bilaga är att framföra ett individuellt arbete i kursen programvaruutvecklingmetodik (TDDD77) som ges på Linköpings universitet som kandidatprojekt för studenter som läser till civilingenjör inom datateknik. 
\newline
\newline
Denna bilaga skrivs inom området kvalitetssäkring, då min roll i kandidatprojektet har varit kvalitetssamordnare. Som kvalitetssamordnare har mitt arbete bestått av att se till att kod och dokument skrivs efter en viss standard som kommer förse att slutprodukten håller tillräcklig hög kvalité.

\subsection{Frågeställning}

\begin{itemize}
  \item Vad krävs för att en slutprodukt eller tjänst ska vara av tillräckligt god kvalité?
  \item The second item
  \item The third etc \ldots
\end{itemize}

\subsection{Avgränsningar}

Ruben: Kvalitetssäkringen skall stödja teamet att utveckla något som har rätt nivå. I vissa fall är det
felfrihet som gäller, i vissa fall att komma ut med en beta-test före konkurrenterna. Ämnet är intressant, 
speciellt som ni har ett litet ovanligt projekt; Ni säger er inte följa någon egentlig metod, och 
implementationen av algoritmen står i centrum. I många fall är det rätt svårt att testa optimeringsprogram. Så, vinkla ditt bidrag mot just ert projekt, så blir det intressant och inte onödigt 
brett.
