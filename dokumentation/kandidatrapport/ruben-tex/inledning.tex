\section{Inledning}
Kvalitetssäkring evaluerar och modifierar en organisations metoder för att se till att en slutprodukt håller rätt nivå. Vad som anses vara rätt nivå kan skilja sig mycket beroende på kundens behov. 
\newline
\newline
Att lyckas uppnå en god kvalitetsstandard genom alla steg vid utvecklandet av en produkt eller tjänst kräver struktur, en struktur som sätts upp av en enskild individ eller en grupp av personer med relevant kunskap. I denna individuella rapport tänker jag ta upp vad som krävs för att uppnå en sådan struktur som sedan kan hjälpa till att generera en produkt eller tjänst av tillräckligt hög kvalité till kundens behov. Jämförelser gentemot hur kvalitetssäkringen har gått tillväga i ett kandidatprojekt som jag har varit delaktig i.

\subsection{Syfte}
Syftet med denna individuella rapport är att fördjupa mig i ett område som har varit relevant till min roll som kvalitetssamordnare i kandidatprojektet ''Prediktionsreglering''. 
\newline
\newline
Genom att fördjupa mig inom kvalitetssäkring kommer jag kunna reflektera om vad som skulle ha kunna gjorts bättre för projektet jag har varit delaktig i, men också diskutera om svårigheterna som kan uppkomma vid kvalitetssäkringen av ett projekt.

\subsection{Frågeställning}

\begin{enumerate}
  \item Vad är rätt nivå gällande kvalité?
  \item Vad krävs för att en slutprodukt eller tjänst ska vara av tillräckligt god kvalité?
  \item Har slutprodukten i kandidatarbetet tillräckligt god kvalité?
\end{enumerate}

\subsection{Avgränsningar}

Då huvudsyftet i detta kandidatprojekt har varit att implementera en optimeringsalgoritm kommer denna rapport främst fokusera på hur kvalitetssäkringen har gått tillväga för projektet, men även vilka förbättringar som skulle ha kunna gjorts. Detta för att rapporten inte ska bli onödigt bred och för att tiden är begränsad.


