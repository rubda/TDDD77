\section{Kodstandard}

\lstset{ language=C, backgroundcolor=\color{black!5}, basicstyle=\footnotesize}

\begin{itemize}
  \item Indentering sker med två blanksteg.
  \item Funktionsnamn och variabelnamn skrivs med små bokstäver med understreck som separerator mellan ord. Får heta vad författaren önskar så länge det är relevant. 
  \item Pekare skrivs med asterisken direkt efter datatypen.
  \item Filer inkluderas där de behövs, antingen i c- eller h-filen.
  \item Den öppnande måsvingeparentesen skrivs på samma rad som funktionsnamnet och returtypen. Den stängande parentesen skrivs ensam på raden efter den avslutande satsen i funktionen. If-satser och liknande skrivs på samma sätt.
  \item Typedef sker separat från struct-deklarationer.
  \item Kommentarer skrivs som i exemplet nedan.
  \item Koden skall vara skriven på engelska.
\end{itemize}

\begin{lstlisting}
/*
  Author: Alexander Yngve
  Date: 2015-02-11
  Description: Short description of why this file is needed.
*/

#include <stdio.h>

/* Short explanation of why this type is needed. */
struct struct_t{
  int* pointer;
  int number;
};

typedef struct struct_t struct_t;

/* Short explanation of why this function is needed. */
int main(){
  printf("Hej\n");

  struct struct_t data_1;
  data_1.number = 4;
  printf("%i\n", data_1.number);

  struct_t data_2 = {NULL, 5};
  printf("%i\n", data_2.number);

  return 0;
}
\end{lstlisting}

