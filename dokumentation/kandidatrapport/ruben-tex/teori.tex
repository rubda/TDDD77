\section{Teori}
I boken ''Software Product Quality Control'' \cite{SPQC} nämns ett antal definitioner som förtydligör vad kvalitetssäkring innebär, dessa syns nedan.  

\begin{itemize}
  \item Quality assurance: a planned and systematic pattern of all actions necessary to provide adequate confidence that an item or product conforms to established technical requirements. 
  \item Constructive quality assurance: All means to be used in constructing a product in a way so it  meets its quality requirements. 
  \item Analyctical quality assurance: All means of analysing the state of the quality of a product. 
\end{itemize}
\noindent Kvalitetssäkring innebär följaktligen att man som leverantör av en produkt eller tjänst ska se till att de uppfyller de krav som har satts upp i en eventuell kravspecifikation. Att man under arbetsgången analyserar om man är på väg att uppfylla kraven eller inte, isåfall måste detta åtgärdas omedelbart.
\newline
\newline
Med detta sagt finns det flera steg i ett projekt att kvalitetssäkra. Ett effektivt sätt att göra detta på är genom att följa Shewhart cykeln, det vill säga planera, göra, studera och agera (PGSA). Shewhart cykeln är en iterativ fyrsteg metod, se figur \ref{fig:shewcycle}.
\newline

\begin{figure}[h]
\centerline{\includegraphics[scale=0.5]{ruben-tex/graphic/shewhartcycle}}
\caption{Shewhart cykeln}
\label{fig:shewcycle}
\end{figure}

\begin{enumerate}
  \item Planera. I detta skede ska målet fastställas, det vill säga sätta upp de krav som behövs för att kunden skall bli nöjd. Genom att göra detta är det tydligt om vad som skall göras och en överenskommelse finns mellan kund och leverantör. 
  \item Göra. När man väl har fått för sig om vad som behövs göras för att kunden skall bli nöjd är det dags att implementera ett sätt att jobba och fullfölja processen.
  \item Studera. Efter att ha följt processen under en viss period är det dags att utvärdera om processen man har följt kommer leda till att man uppfyller de målen man har fastställt i ''planera''-skedet.
  \item Agera. Om man under ''studera''-skedet upptäcker att processen man följer inte kommer leda till att man uppfyller de krav som kund och leverantör var överens om måste detta åtgärdas omedelbart genom att planera om arbetsprocessen eller i viss mån diskutera kraven med kunden. Om processen som följs kommer uppfylla de krav som har satts upp kan man forsätta med iterationerna av Shewhart cykeln precis som innan.
\end{enumerate}

\begin{figure}[h]
\centerline{\includegraphics[scale=0.15]{ruben-tex/graphic/PDCA_Process}}
\caption{PDCA process}
\label{fig:pdcaprocess}
\end{figure}

\noindent Genom att följa Shewhart cykeln under ett projekt kan man iterativt förbättra sin arbetsprocess och genom detta öka kvaliteten av den produkt man utvecklar. Se figur \ref{fig:pdcaprocess}.
\newline
\newline
När det väl är dags för överlämning av en produkt eller tjänst är det bra att göra en kravinspektion innan. En kravinspektion är ''A process or meeting during which a software product is examined by a project personnel, managers, users, customers, user representatives, or other interested parties for comment or approval'' \cite{SFSR}. Det man vill få gjort med en kravinspektion är alltså att produkten eller tjänsten som har tagits fram uppfyller de krav som har fastställts i början av arbetet. De berörda parterna ska vara nöjda med det som har åstadkommits.



