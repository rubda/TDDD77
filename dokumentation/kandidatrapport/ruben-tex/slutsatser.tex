\section{Slutsatser}
Syftet med detta bidrag var att fördjupa sig inom kvalitetssäkring, för att sedan kunna reflektera om vad som kunde ha gjorts bättre för det projektet jag har varit delaktig i.
\newline
\newline
För det första har jag lyckats att fördjupa mig inom kvalitetssäkring, under kapitel ett hade jag ställt ett par frågor som skulle få svar genom att läsa på om vad kvalitetssäkring innebär. Under kapitel 3 har jag tagit lärdomar om vad kvalitetssäkring innebär och hur man kan uppnå en produkt av god kvalité. Sammanfattningsvis kan man säga att kvalitetssäkring innebär att man som leverantör uppfyller de krav en kund har gett en och att man under arbetsgången analyser om man är på väg att uppfylla kraven eller inte. Metoder som att använda Shewhart cykeln finns tillhands för att göra detta. Avslutningsvis kan man göra en kravinspektion för att verkligen fastställa med alla berörda parter att slutprodukten gör det den ska.
\newline
\newline
Slutligen så anser jag inte att optimeringsalgoritmen som skapades är av god kvalité då vi inte lyckades uppfylla alla de krav som vi hade satt upp från början. Optimeringsalgoritmen skulle på förhand ha en exekveringstid likvärdig mot mjukvaran ''Gurobi'' som är en kommersiell mjukvara enbart för optimeringsproblem. Detta var inget vi lyckades med och anledningen till att vi inte lyckades med detta är brist på tid och kunskap skulle jag påstå. En bättre struktur på kvalitetssäkringen under projektet skulle kunna ha gett oss mer tid, då som tidigare nämnt var en svårighet under projektet att kodstandarden inte följdes helt och en analys på om vi var på väg att möta kraven eller inte gjordes alldeles för få gånger.
