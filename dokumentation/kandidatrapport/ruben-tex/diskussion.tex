\section{Diskussion}
Under denna del kommer en diskussion angående om resultaten ske samt metoderna som användes.
\subsection{Resultat}
Resultaten för alla tre punkter måste anses vara rimliga. Den första punkten angående vad som är rätt nivå är en väldigt diffus fråga och kan bara svaras genom att båda parter, dvs leverantör och kund har ett enhetligt svar. Eftersom det man kommer överens om i en eventuell kravspecifikation måste räcka att fullfölja för att det ska klassas som rätt nivå gällande kvalité. Sen om kunden hade velat haft några andra funktioner med produkten eller tjänsten så borde kunden ha specificerat detta. Leverantörer har också skyldighet att försöka uppfylla alla krav som har specificerats på ett snyggt sätt och inte bara slänga ihop något som fungerar, små detaljer kan utmärka ett arbete.
\newline
\newline
Den andra punkten finns det inte mycket att diskutera om, som allt annat i livet krävs det struktur för att man ska uppnå något. Under teoridelen i kapitel tre kan man konstatera att planera, utföra och granska antagligen kommer generera en produkt eller tjänst av tillräckligt god kvalité.
\newline
\newline
Angående den sista punkten kan man byta svaret från ett ja till ett nej, då ett krav förhandlades bort. I detta projekt hade kandidatgruppen endast ett mätbart krav, detta krav specificerade att optimeringsalgoritmen skulle vara likvärdig i hastighet i jämförelse med en kommersiell mjukvara kallad ''Gurobi''. ''Gurobi'' specialiserar sig i att lösa olika sorters optimeringsproblem. Under projektets slutskede diskuterade kandidatgruppen med Daniel Simon om att det kommer bli svårt att vara jämbördig med ''Gurboi'', detta var inget problem för Daniel Simon och kravet kunde tas bort. Kontaktpersonen tryckte på att det viktigaste var att koden funkar och att den är väldokumenterad, vilket den är. Däremot kan jag tycka att om man tar bort ett krav så har man misslyckats och därför anser jag att man kan ändra svaret till nej, slutprodukten håller inte tillräckligt god kvalité, men då är jag väldigt hård mot mig själv och kandidatgruppen. Däremot säger teorin bakom det hela att produkten håller tillräckligt god nivå då den uppfyller alla krav, det ska inte spela någon roll om man har förhandlat bort något krav eller inte.

\subsection{Metod}
Efter slutförandet av projektet och analys av hur det hela gick finns det många saker man hade kunnat gjort annorlunda. Ett återkommande problem som stöttes på under projektets gång var att inte många respekterade kodstandarden som hade satts från början, inte förens den sista iterationen började folk ta den på allvar. Detta resulterade i att mycket tid lades ner på att refaktorera kod. Kodstandarden innehöll inte heller allt som bör ta hänsyns till, men som bör vara uppenbara för en programmerare. Ett exempel är att frigöra minne, det gjordes knappt och det resultera i att vissa var tvungna att hitta minnesläckorna och åtgärda dem. Kandidatgruppen hade ett krav på att skriva pålitlig kod och minnesläckor i koden är inte pålitligt.
\newline
\newline
Jag tror att slutprodukten skulle ha sett annorlunda ut om kodstandarden täckte fler områden i hur man ska skriva kod samt om man tryckte på hur viktigt det är att följa kodstandarden. Eftersom kandidatgruppen är en grupp av klasskamrater är det inga seriösa arbetsförhållanden mellan individerna. Detta leder till att kod bara slängs ihop och man tar inte kodstandarden seriös, jag som kvalitetssamordnare hade kunnat trycka mer på att kodstandarden skall följas för att en produkt av bättre kvalité skulle skapats.
\newline
\newline
I teoridelen visade forskningen att om man ska uppnå en högkvalitativ produkt kan man t.ex. använda sig av Shewhart cykeln, planera, göra, studera och agera. Kandidatgruppen planera hur man skulle skriva kod, man följde inte kodstandarden, kandidatgruppen var väl medveten om det. Vid detta stadiet borde man agera, visserligen sas det åt folk att börja använda kodstandarden, men det hade nog varit bättre att utvärdera om kodstandarden och föra dialog om varför den inte följs för att skriva en ny som möjligtvis alla känner sig bekväma med.
