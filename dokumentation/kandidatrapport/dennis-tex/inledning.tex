\section{Inledning}
Teknisk dokumentation används för att beskriva tekniska eller specialiserade ämnen och göra informationen mer förståelig och användbar för läsarna. I sådana dokument är vetenskapliga formler och komplexa notationer ett vanligt förekommande och behöver skrivas på ett speciellt sätt. 
\newline
\newline
På slutet av 1970-talet utvecklade Donald E Knuth ett typsättningssystem kallat {\TeX} speciellt för komplexa matematiska formler. {\LaTeX} släpptes år 1994 av Leslie Lamport som en förlängning av {\TeX} som gav utökad funktionalitet för att göra dokument. \citep{latexandfriends}  

\subsection{Syfte}
Syftet med denna rapport är att klargöra för- och nackdelar med programmeringsspråket {\LaTeX} och hur väl det lämpar sig för teknisk dokumentation i ett programmeringsprojekt. 

\subsection{Frågeställning}
\begin{itemize}
\item Lämpar sig {\LaTeX} för teknisk dokumentation i ett programmeringsprojekt? 
\item Är {\LaTeX} värt att lära sig i fråga om tid?
\end{itemize}
		
\subsection{Avgränsningar}


	