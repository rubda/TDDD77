\section{Inledning}
Teknisk dokumentation används för att beskriva tekniska eller specialiserade ämnen och göra informationen mer förståelig och användbar för läsarna. I sådana dokument är vetenskapliga formler och komplexa notationer ett vanligt förekommande och behöver skrivas på ett speciellt sätt. Då dokumentation idag sker digitalt ställer det allt högre krav på textredigerare och dylikt som har många olika inställningar och
\\
På 1970-talet utvecklade Donald E Knuth ett typsättningssystem speciellt för komplexa matematiska formler. Detta språk heter Tex


\subsection{Syfte}
Syftet med denna rapport är att klargöra för- och nackdelar med programmeringsspråket {\LaTeX} och hur väl det lämpar sig för teknisk dokumentation.  

\subsection{Frågeställning}
\begin{itemize} 
\item Lämpar sig Latex bättre för teknisk dokumentation än textredigerare som Word?
\end{itemize}
		
\subsection{Avgränsningar}


	