\section{Inledning}
Teknisk dokumentation används för att beskriva tekniska eller specialiserade ämnen och göra informationen mer förståelig och användbar för läsarna. I sådana dokument är vetenskapliga formler och komplexa notationer ett vanligt förekommande och behöver skrivas på ett speciellt sätt. Då dokumentation idag främst sker digitalt ställer det allt högre krav på textredigerare och dylikt som har många olika inställningar och
\\
På slutet 1970-talet utvecklade Donald E Knuth ett typsättningssystem speciellt för komplexa matematiska formler. Detta språk heter {\TeX}
\\
{\LaTeX} skrevs av Leslie Lamport som en förlängning av {\TeX} språket. Det är ett turing-komplett språk och typsättningssystem.  


\subsection{Syfte}
Syftet med denna rapport är att klargöra för- och nackdelar med programmeringsspråket {\LaTeX} och hur väl det lämpar sig för teknisk dokumentation.  

\subsection{Frågeställning}
\begin{itemize} 
\item Lämpar sig {\LaTeX} bättre för teknisk dokumentation än textredigerare som Word?
\item  
\end{itemize}
		
\subsection{Avgränsningar}


	