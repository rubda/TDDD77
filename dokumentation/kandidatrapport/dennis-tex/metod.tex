\section{Metod}  

\subsection{{\LaTeX} i dokumenten}
De huvudsakliga funktionaliteterna för dokumenten som har testats är: 
\begin{itemize}
\item Dokumentstruktur
\item Infoga bilder
\item Matematiska formler och algoritmer
\item Referenslistor och källhänvisningar
\end{itemize}   
För själva strukturen av dokumentet användes en färdig mall som innehöll kommandon för att göra ett dokument enligt LIPS-modellen \citep{lips}. Det behövdes dock göras några egna kommandon för att få den önskade slutliga strukturen.
\newline
\newline
När bilder skulle infogas i dokumenten användes miljön ''figure''. Oftast användes parametern ''h'' för att få figuren att hamna ungefär där den infogades i texten. Dock om det inte fanns tillräckligt med plats på sidan kunde figuren hamna på otympliga ställen. Då kunde parametern ''t'' användas för att tvinga figuren till överst på sidan eller parametern ''H'' för att tvinga figuren att hamna där den infogades.   
\newline
\newline
När matematiska formler behövdes i dokumenten användes {\LaTeX} inbyggda kommandon för detta. Detta räckte för att beskriva de ekvationer som behövdes. Miljön ''equation'' användes nästan alltid för att få ekvationen att framstå tydligare och kunna refereras till. 
\newline
\newline
För referenslistor och källhänvisningar användes ''natbib'' och     
\subsection{{\LaTeX} i gruppen}
För att analysera hur väl {\LaTeX} har fungerat i gruppen har under projektets gång följande faktorer observerats:   
\begin{itemize}
	\item Hur väl gruppens medlemmar lyckats behärska {\LaTeX} under tiden som har gått:
	\begin{itemize}
		\item Med avseende på hur de själva tycker att de behärskar {\LaTeX}. 
	\end{itemize}
	\item Hur användet av {\LaTeX} på olika operativsystem påverkat effektiviteten.
\end{itemize} 
För att utföra denna studie har gruppens medlemmar använt {\LaTeX} på Linux, Windows och Mac. De har även intervjuats under projektets gång.   
