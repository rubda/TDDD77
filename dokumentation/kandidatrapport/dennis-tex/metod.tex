\section{Metod}  
Nedanför beskrivs metoderna relevanta för rapporten.  

\subsection{{\LaTeX} i dokumenten}
De huvudsakliga funktionaliteterna för dokumenten som har testats är: 
\begin{itemize}
\item Dokumentstruktur
\item Infoga bilder
\item Matematiska formler och algoritmer
\end{itemize}   
För själva strukturen av dokumentet användes en färdig mall som innehöll kommandon för att göra ett dokument enligt LIPS-modellen \citep{lips}. Det behövdes dock göras några egna kommandon för att få den önskade slutliga strukturen. 
\newline
\newline
När bilder skulle infogas i dokumenten användes miljön ''figure''. Oftast användes parametern ''h'' för att få figuren att hamna ungefär där den infogades i texten. Om det dock inte fanns tillräckligt med plats på sidan kunde figuren hamna på otympliga ställen. Då kunde parametern ''t'' användas för att tvinga figuren till överst på sidan eller parametern ''H'' för att tvinga figuren att hamna där den infogades.
\newline
\newline
När matematiska formler behövdes i dokumenten användes {\LaTeX} inbyggda kommandon för detta. Detta räckte för att beskriva de ekvationer som behövdes. Miljön ''equation'' användes nästan alltid för att få ekvationen att framstå tydligare och kunna refereras till. Algoritmer infogades antingen direkt från deras källkodsfil med kommandon från paketet ''listings'' eller genom pseudokod med kommandon från paketen ''algorithm'' och ''algpseudocode''.  
     
\subsection{Inlärning av {\LaTeX}}
Under projektets gång har det analyserats hur väl gruppens medlemmar lyckats behärska {\LaTeX}. Detta har gjorts med avseende på hur de själva tycker att de behärskar det.      

\subsection{{\LaTeX} på olika plattformar}
{\LaTeX} har under projektet använts på plattformarna Windows, Mac och Linux. Olika textredigeringsprogram har använts på respektive plattform. På Windows och Mac har TexMaker använts och på Linux har Gummi använts. Det har analyserats hur väl det har fungerat med avseende på funktionaliteten och användbarheten hos de olika programmen och om det har lett till onödig tidsåtgång.   
