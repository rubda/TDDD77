\section{Standarder}

\subsection{Syfte}
Denna del av dokumentet är till för att klargöra vilka standarder som bör följas för att uppnå en god kodkvalité.

\subsection{Innehåll}
\begin{itemize}
\item Skrivning av kod
\item Testning av kod
\end{itemize}

\subsubsection{Kod}
Källkod för optimeringsalgoritmen skall vara skriven i språket C99 och följa standarden som tas upp i bilaga A. Även kommentering av kod ska följa standarden som tas upp i bilagan. 
\newline
\newline
Koden ska kompileras med strängen som syns nedan. Filerna kommer kompileras med gcc 4.8. Flaggorna som har valts är för att få en väldigt sträng ``rättning`` av koden som skrivs. 

\begin{itemize}
\item gcc -Wall -pedantic -std=c99 -o namn namn.c
\end{itemize}

Källkod för GUI:t ska skrivas i Python efter PEP 8 och PEP 257 standarderna. 

\subsubsection{Testning}
Testning ska ske efter riktlinjerna som finns i ``Testplan", [Isaksson, 2015]. 
