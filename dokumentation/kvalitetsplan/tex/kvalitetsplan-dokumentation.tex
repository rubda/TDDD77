\section{Dokumentation}

\subsection{Syfte}
Denna del kommer att beskriva de nödvändiga dokumenten för projektet. Syftet med dokumentationen är att möjliggöra förståelse för projektet.

\subsection{Dokumentlayout och konsistens}
Dokumenten skall vara skriva på svenska. De ska även vara skrivna i latex, men slutprodukten ska vara i pdf-format. Varje dokument ska följa en specifik mall, vare sig LIPS-mallar eller standard IEEE-mallar. 

\subsection{Minimikrav för dokumentation}
\begin{itemize}
\item Kravspecifikation
\item Projektplan
\item Kvalitetsplan
\item Utvärdering av förstudien
\item Arkitekturdokument
\item Testplan
\item Teknisk dokumentation
\item Användarhandledning
\end{itemize}

\subsubsection{Kravspecifikation}
Kravspecifikationen består främst av de krav som ställs på produkten som ska tas fram. Även produktens syfte samt mål. Dokumentet heter "Kravspecifikation", [Ljung, 2015].

\subsubsection{Projektplan}
Projektplanen består främst av aktiviteter, milstolpar och organisationsplan. Dokumentet heter "Projektplan", [Ljung, 2015].

\subsubsection{Kvalitetsplan}
Kvalitetsplan är detta dokument och ska ge övergripande riktlinjer för de kvalitetskriterier som skall följas av projektmedlemmarna. [Das, 2015].

\subsubsection{Utvärdering av förstudien}
Detta dokument innehåller en utvärdering av förstudien. Detta dokument är ej påbörjad, då förstudien inte är klar. 

\subsubsection{Arkitekturdokument}
Detta dokument ska innehålla en övergripande design över hur produkten ska se ut i form av figurer och pseudokod. Artitekturdokumentet uppdateras löpande just nu. Dokumentet heter ``Arkitektur", [Fast, 2015].

\subsubsection{Testplan}
Testplanen innehåller testning som ska genomföras under projektets iterationer. Dokumentet heter ``Testplan", [Isaksson, 2015].

\subsubsection{Teknisk dokumentation}
Detta dokument kommer innehålla en övergripande dokumentation över hur systemet är uppbyggt och hur det funkar. Detta dokument är ej påbörjat. 

\subsubsection{Användarhandledning}
Användarhandledningens innehåll kommer bestå av hur produkten används. Detta dokument är ej påbörjat.

\subsection{Övrig dokumentation}

\subsubsection{Mötesrapporter}
Mötesrapporter ska innehålla datumet när mötet hölls, punkter som diskuterades och även vad som kom fram av diskussioner i små drag.

\subsubsection{Tidsrapporter}
Tidsrapporter ska fyllas i senast varje söndag tills projektavslut. Varje projektmedlem ska även skriva in antalet timmar de har jobbat med en viss uppgift. Denna rapport ska skickas till handledaren.

\subsubsection{Statusrapporter}
Statusrapporter ska innehålla projektgruppens status om hur arbetet går, vilka problem som har uppstått och vad man har lyckats med. Denna rapport ska skickas till handledaren och även till kunden om kunden begär det. 
