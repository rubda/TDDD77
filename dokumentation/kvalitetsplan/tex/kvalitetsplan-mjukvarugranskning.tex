\section{Granskningar}
Denna del av dokumentet avser de granskningar som ska göras på komponenterna som utgör slutprodukten. 

\subsection{Minimikrav}
Under projektets gång kommer ett visst antal granskningar göras, de viktigaste nämns i denna del av dokumentet. 

\subsubsection{Kravgranskning}
Kravgranskning skall göras för att säkerhetsställa att kraven i kravspecifikationen är tillräckliga samt realiserbara.

\subsubsection{Arkitekturgranskning}
Arkitekturgranskning skall göras för att undersöka om den uppfyller det som har sagts i kravspecifikationen. Granskningen ska även se till att arktitekturen håller hög kvalité samt är realiserbar inom tidsramen.

\subsubsection{Slutgranskning}
Slutgranskning innefattar att slutprodukten granskas så att den uppfyller det som har specifierats i kravspecifikationen samt artitekturdesignen. 

\subsubsection{Dokumentgranskning}
Alla dokument som produceras av projektgruppen ska granskas för att undersöka om de håller riktlinjerna som har nämnts i detta dokument, d.v.s undersöka om de håller en god nivå. 



