\section{Matlab}
Beskrivning för installation och användande av QuadOpt i Matlab.

\subsection{Installera QuadOpt i Matlab}
\begin{enumerate}
	\item Se till att du har en C-kompilator installerad.
	\item Packa upp \emph{quadopt.zip} i ditt Matlab workspace.
	\item Kör nu skriptet \textbf{build} i Matlab.
	\item Klart! QuadOpt är nu installerat i Matlab och är redo att köras.
\end{enumerate}

\subsection{Användning}
Anropa QuadOpt lösaren i Matlab genom att skriva

\begin{lstlisting}
z = quadopt(Q, q, E, h, F, g, iter, time);
\end{lstlisting}

\begin{itemize}
	\item z - Matris där lösningen sparas.
	\item Q - Matris innehållande den kvadratiska delen av problemet.
	\item q - Matris innehållande den linjära delen av problemet.
	\item E - Ekvivalensbivillkorsmatrisen.
	\item h - Ekvivalensbivillkoren i högerledet.
	\item F - Olikhetsbivillkorsmatrisen (Obs att det är större än eller lika med g).
	\item g - Olikhetsbivillkoren i högerledet.
	\item iter - Max antal iterationer innan lösaren avbryter.
	\item time - Max tid i mikrosekunder innan lösaren avbryter.
\end{itemize}
