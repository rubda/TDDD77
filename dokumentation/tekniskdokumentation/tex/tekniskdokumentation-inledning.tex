\section{Inledning}
Den tekniska dokumentationen är skriven i syfte att vara till hjälp för användande, utveckling, felsökning och underhåll av programvaran QuadOpt, som används i huvudsak till lösning av konvexa kvadratiska optimeringsproblem, utvecklad av grupp 2 i kursen Kandidatprojekt i mjukvaruutveckling (TDDD77) vårterminen 2015 vid Tekniska högskolan vid Linköpings universitet.

(OBS lägg till doxygen dokumentet som bilaga)

\subsection{Parter}
Kunden i detta projekt är Daniel Simon som är industridoktorand på Linköpings universitet och anställd av Saab AB. Projektgruppens medlemmar består av studenter från linjen datateknik vid LiTH, handledare är Andreas Runfalk och examinator för kursen är Kristian Sandahl.



%\subsection{Leverans}
%Till kunden levereras källkod och binärer för solvern, GUI:t, 

%med tillhörande källkod och binärer, samt dokumentation för dessa. Kunden erhåller även den tekniska dokumentationen för vidareutveckling och användning av QuadOpt.

%Anledningen till att QuadOpt utvecklas är för att kunden ville ha ett eget program som han kunde ändra källkoden och skriva om funktioner, de komersiella programmen tillåter inte alltid detta pga olika licenser och man får ofta inte tillgång till källkoden.möjliggöra 