\section{Systemdelar}
QuadOpt innehåller ett flertal olika moduler som används för att lösa och tolka optimeringsproblemet, förklaringar av hur dessa är uppbyggda och fungerar kommer att ges i detta avsnitt.
(OBS kan möjligtvis vara bra att ha egna filer för dessa underrubriker om det blir mycket)

\subsection{Lösaren}

\begin{algorithm}[H]
\caption{Active set method}
\label{alg:activeset}
\begin{algorithmic}
\Procedure{Active set method}{}
\State Compute a feasible starting point $x_0$;
\State Set $W_0$ to be a subset of the active constraints at $x_0$;
\For{$k = 0, 1, 2,...$}
	\State Solve subproblem to find $p_k$;
	\If{$p_k = 0$}
		\State Compute Lagrange multipliers $\hat{\lambda_i}$,
		\State set $\hat{W} = W_k$; 
		\If{$\hat{\lambda_i} \ge 0$ for all $i \in W_k \cap I$}
			\State \textbf{STOP} with solution $x^* = x_k$;
		\Else
			\State Set $j =$ argmin$_{j \in W_k \cap I}\hat{\lambda_j}$;
			\State $x_{k+1} = x_k; W_{k+1} \gets W_k$ \textbackslash \{$j$\};		
		\EndIf
	\Else		
		\State Compute $\alpha_k$ from stepformula;
		\State $x_{k+1} \gets x_k + \alpha_k p_k$;
		\If{There are blocking constraints}
			\State Obtain $W_{k+1}$ by adding one of the blocking constraints to $W_{k+1}$;
		\Else
			\State $W_{k+1} \gets W_k$;	
		\EndIf 	
	\EndIf
\EndFor 
\EndProcedure
\end{algorithmic}
\end{algorithm}

\subsection{MATLAB - MEX} \label{subsec:mex}

MEX står för \textbf{M}atlab \textbf{ex}ecutable är utvecklat av MathWorks och används för att bygga MATLAB funktioner från C/C++ och Fortran funktioner. Det innehåller ett bibliotek med funktioner för att konvertera och skicka datatyper mellan MATLAB och C. För att kunna använda en C funktion i MATLAB behöver en mexfunktion användas i C filen, se figur~\ref{fig:mex2}. 

\begin{figure}[H]
\lstinputlisting[language=C]{tex/mex.c}
\caption{MEX gateway routine}
\label{fig:mex2}
\end{figure}  

Denna funktion ger tillgång till inskickade objekt från MATLAB i fältet ''prhs[]'' och utgående objekt ska läggas i fältet ''plhs[]''. Dessa objekt är av typen ''mxArray'' vilket är en datatyp som används av MATLAB. Med hjälp av olika funktioner i MEX-biblioteket kan dessa datatyper konverteras till datatyper som C kan använda och tillbaka. Några händiga funktioner följer:
\begin{itemize}
\item mxGetM(mxArray) - returnerar mxArray rader
\item mxCreateDoubleMatrix(rader, kolumner, mxREAL) - returnerar en mxArray
\end{itemize}



\subsection{Matlib}
%Förklaring av viktiga saker i matrisbiblioteket, så det kan vidareutvecklas. beskrivningar av funktioner kommer finnas i doxygen dokumentet.
Matrisbiblioteket innehåller alla de matrisoperationer som QuadOpt kan tänkas behöva, där många utav dessa har blivit optimerade för att få ut bästa prestanda på de storlekar av matriser som det är tänkt att QuadOpt kommer att arbeta på. Nedan syns en C-kod implementation av hur matrisernas grundläggande datastruktur ser ut. När operationer utförs på matriserna så är det matrisernas pekare som skickas... optimera prestandan ... FYLL I HÄR

\lstinputlisting[language=C]{tex/matrix_struct.c}
\subsection{Byggsystem}
Ett byggsystem används för att optimera utvecklandet av QuadOpt, där hela systemets delar byggs tillsammans med ett kommando för att man på ett snabbare och enklare sätt ska se om allt fungerar....  FYLL I

\subsection{Parser}
Parsern är ett alternativ som möjliggör för användaren att generera en C-fil med vilken lösaren löser optimeringsproblemet. För att generera C-filen krävs det att parsern får två olika filer, en datafil med matrisdata och en fil med optimeringsproblemet där bland annat matrisernas dimensioner definieras. Dessa båda filer skickas till parsern från GUI:t. Anledning till parsern är att underlätta för användaren då många matriser som lösaren använder sig av är väldigt stora och innehåller många nollor. Parsern tar istället in mindre matriser som innehåller den data som dessa större matriser innehåller, bortsett från nollorna.