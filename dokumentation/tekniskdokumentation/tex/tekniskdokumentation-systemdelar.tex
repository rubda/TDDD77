\section{Systemets moduler}
QuadOpt innehåller ett flertal olika moduler som används för att lösa och tolka optimeringsproblemet, förklaringar av hur dessa är uppbyggda och fungerar kommer att ges i detta avsnitt.

\subsection{Solvern}
Alla filer som rör solvern finns i mappen ''quadopt'' under projektets rotmapp.

\subsubsection{Datatyper}
Filen ''src/problem.c'' och ''include/problem.h'' definierar datatypen ''problem'' samt funktioner för hantering av denna. Den innehåller all information som definierar optimeringsproblemet, matriser, antal variabler, antal bivillkor samt externa iteration- och tidsbegränsningar. Datatypen ''problem'' används av algoritmen beskriven i avsnitt \ref{sec:algoritm} samt hjälpfunktionerna beskrivna i avsnitt \ref{sec:helpfun}.
\newline
\newline
Datatypen ''work\_set'' definieras i ''include/work\_set.h'' och dess tillhörande funktioner i ''src/work\_set.c''. Den beskriver en mängd av heltal och används av algoritmen för att hålla koll på aktiva bivillkor. Den används även för att hålla kolla på rader i simplextablån när startpunkten beräknas, detta finns beskrivet i avsnitt \ref{sec:helpfun}.

\subsubsection{Algoritm} \label{sec:algoritm}
 I algoritm \ref{alg:quadoptsolver} finns på pseudokodsform en beskrivning av hur solvern löser problemet med Active set metoden. Algoritm \ref{alg:quadoptsolver} finns implementerad i filen ''src/solver.c''.

\begin{algorithm}[H]
\begin{algorithmic}
\Procedure{QuadOpt-solver}{$problem$ $P$}
\If{$P$ has not feasible starting point $z_0$}
	\State Compute a feasible starting point $z_0$;
\EndIf	
\State Set $activeSet$ to be a subset of the active constraints at $z_0$ in $P$;
\While{$\textbf{true}$}
	\State Solve subproblem to find step direction $p$;
	\If{$p$ is zero vector}
		\If{$activeSet$ has zero constraints}
			\State \textbf{break};
		\EndIf		
		\If{Could not remove constraints from $activeSet$}
			\State \textbf{break}
		\EndIf
	\Else
		\State Take step to better feasible point $z$ in $P$;
		\If{Could not step}
			\State \textbf{break};
		\EndIf
		\State Set $activeSet$ with new active constraints at $z$;	
	\EndIf
\EndWhile
\State  $solution$ in $P\gets z$;
\State \textbf{return} $solution$ in $P$;
\EndProcedure
\end{algorithmic}
\caption{QuadOpt-solver}
\label{alg:quadoptsolver}
\end{algorithm}

\noindent Pseudokod för hur subproblemet i algoritm \ref{alg:quadoptsolver} löses finns i algoritm \ref{alg:subproblem}

\begin{algorithm}[H]

\begin{algorithmic}
\Procedure{Subproblem}{$problem$ $P$}
\If{$activeSet$ has zero constraints}
\State set $p$ pointing towards global optimum;
\Else
\While{number of constraints in $activeSet$ $\geq$ number of variables}
\State remove constraint with most negative Lagrange multiplier;
\If{could not remove constraint}
\State set $p$ to zero vector;
\State Return;
\EndIf
\EndWhile
\State use ''Range-space'' to find $p$;
\EndIf
\EndProcedure
\end{algorithmic}

\caption{Subproblem}
\label{alg:subproblem}
\end{algorithm}

\noindent ''Range-space'' innehåller ett antal matrisoperationer för att ta reda på riktningsvektor $p$. Teorin bakom denna metod kan läsas om i Numerical Optimization av J Nocedal och S Wright.

\subsubsection{Hjälpfunktioner} \label{sec:helpfun}
I filen ''src/trans\_con.c'' finns funktioner för att göra om ett problem på parser-form till solver-form. Den används av parsern när problemet har matats in i GUI:t och C-kod ska genereras.
\newline
\newline
I filen ''src/simplex.c'' finns funktioner för att lösa ett linjärt problem med simplexmetoden. Det finns även hjälpfunktioner för att bygga upp en simplextablå så att en giltig startpunkt till solvern kan tas fram. 

\subsection{MATLAB - MEX} \label{subsec:mex}
All kod rörande MATLAB-gränssnittet ligger i mappen ''matlab'' under projektets rotkatalog. För att kompilera MATLAB-gränssnittet körs skriptet build.m i MATLAB. Om en ny fil läggs till i solvern (i ''quadopt/src'') måste build.m uppdateras. Innan build.m körs måste matrisbiblioteket kompileras.

MEX står för \textbf{M}atlab \textbf{ex}ecutable är utvecklat av MathWorks och används för att bygga MATLAB funktioner från C/C++ och Fortran funktioner. Det innehåller ett bibliotek med funktioner för att konvertera och skicka datatyper mellan MATLAB och C. För att kunna använda en C funktion i MATLAB behöver en mexfunktion användas i C filen, se figur~\ref{fig:mex2}. 

\begin{figure}[H]
\lstinputlisting[language=C]{tex/mex.c}
\caption{Exempel på gränssnitt mellan MATLAB och C}
\label{fig:mex2}
\end{figure}  

\noindent Denna funktion ger tillgång till inskickade objekt från MATLAB i fältet ''prhs[]'' och utgående objekt ska läggas i fältet ''plhs[]''. Dessa objekt är av typen ''mxArray'' vilket är en datatyp som används av MATLAB. Gränssnittskodens uppgift är att konvertera MATLAB-datatyper till solver-datatyper och anropa solvern. Resultatet konverteras sedan tillbaka till MATLAB-datatyper.
\newline
\newline
\textbf{OBS!} All kod som berörs av MATLAB-delen måste vara skriven i C89-standard. I regel måste all kod i quadopt.c filen samt all kod i quadopt-mappen vara i skriven i C89-standard. 


\subsection{Matlib}
Alla filer rörande matrisbiblioteket finns i mappen ''matrixlibrary'' under projektets rotmapp.
\newline
\newline
Matrisbiblioteket innehåller alla de matrisoperationer som QuadOpt kan tänkas behöva. Nedan syns en C-kod implementation av hur matrisernas grundläggande datastruktur ser ut. Matriserna är sparade i row-major ordning.

\lstinputlisting[language=C]{tex/matrix_struct.c}

\noindent Det finns även en till matrisstruktur som används för att spara glesa matriser. Detta format använder formatet ''Coordinate list'' (COO) som innebär att alla nollskilda element, samt deras position i matrisen, sparas i tre fält. Ett fält för värdet, ett fält för deras rad-koordinat och ett sista fält för deras column-koordinat.

\subsection{Byggsystem}
Byggsystemet är skrivet i Make och består av en huvudfil och sex underfiler enligt strukturen i figur \ref{fig:byggsystem}. De viktiga filerna för att bygga kod är huvudfilen i rotkatalogen, ''quadopt/Makefile'', ''matrixlibrary/Makefile'' och ''matlab/Makefile''. Resterande filer används enbart för att bygga dokumentation. För att bygga allting krävs gcc, MATLAB, pdflatex samt en gedigen \LaTeX-distribution.
\newline
\newline
De anrop som kan köras från huvudfilen är följande:
\begin{itemize}
  \item \textbf{all} - bygg dokumentation, matrisbibliotek och quadopt samt kör tester
  \item \textbf{clean} - rensa alla kataloger från byggresultat
  \item \textbf{docs} - bygg all dokumentation
  \item \textbf{libmatrix} - bygg matrisbiblioteket och kör tester
  \item \textbf{quadopt} - bygg quadopt och kör tester
  \item \textbf{test} - bygg matrisbiblioteket, quadopt och kör tester
  \item \textbf{matlab} - bygg MATLAB-tillägg
  \item \textbf{package} - bygg dokumentation, matrisbibliotek, quadopt, kör tester och paketera allt tillsammans med GUI och källkod till MATLAB-tillägget i en zip-fil för distribution
\end{itemize}

\begin{figure}[H]
  \centering
  \begin{verbatim}
\
|-- Makefile
|-- dokumentation
|   |-- Makefile
|   |-- doxygen
|   |   |-- latex
|   |       |-- Makefile
|   |-- kandidatrapport
|       |-- Makefile
|-- matlab
|   |-- Makefile
|-- matrixlibrary
|   |-- Makefile
|-- quadopt
    |-- Makefile
  \end{verbatim}
  \caption{Byggsystemets struktur.}
  \label{fig:byggsystem}
\end{figure}

\noindent Katalogerna ''matrixlibrary'' och ''quadopt'', där den stora mängden kod finns, har en struktur som syns i figur \ref{fig:katalogstruktur}. I mappen ''include''  finns alla headerfiler som definierar gränssnittet mellan källkodsfilerna i mappen ''src'' samt används av testerna i ''src/tests''. Mappen ''obj'' innehåller resultatet av bygget efter att Make har körts.

\begin{figure}[H]
  \centering
  \begin{verbatim}
\
|-- Makefile
|-- include
|   |-- h-filer
|-- obj
|   |-- binärer
|-- src
    |-- c-filer
    |-- tests
        |-- c-filer
  \end{verbatim}
  \caption{Katalogstruktur.}
  \label{fig:katalogstruktur}
\end{figure}

\noindent En ny fil kan läggas till genom att lägga dess c-fil i ''src'', h-fil i ''include'' samt eventuella testfiler (c-filer) i ''src/tests''. De kommer då att hittas automatiskt och kompileras. Testbinärerna kommer även att köras automatiskt vid varje bygge. Om det rör sig om ny funktionalitet i matrisbiblioteket måste en ny regel läggas till i ''matrixlibrary/Makefile'' för att den nya filen ska inkluderas i biblioteket. Detta görs på samma sätt som de två regler som redan finns för filerna ''matLib.c'' och ''sparse\_lib.c''.
\newline
\newline
Om MATLAB-tillägget ska byggas med Make måste variabeln MATLAB\_BIN sättas till sökvägen till MATLAB i filen ''matlab/Makefile''. Tillägget kan också byggas genom att öppna MATLAB och ställa sig i katalogen ''matlab'' samt skriva ''build'' i kommandofönstret.
\newline
\newline
Den Makefile som används av parsern är ''QuadOptGUI/package\_makefile.mk''. När ''make package'' körs från projektets rotkatalog kopieras denna fil till QuadOpts zip-fil.

\subsection{GUI och parser}
QuadOpts GUI är implementerat i Python med hjälp av Tkinter där koden finns i filen ''GUI.py'' där alla grafiska komponenter skapats samt funktioner för den inbyggda texteditorn. Även QuadOpts parser är implementerad i Python och denna ligger i filen ''parser.py'' där alla funktioner för att parsa problemet i problemfilen samt parsa data från en datafil till lösbar C-kod.
\begin{figure}[H]
  \centering
  \begin{verbatim}
parameters
    A (2,2)
    B (2)
    Fx (12,2)
    gx (12)
    P (2,2)
    Q (2,2)
    k (2)
    x_max (2)
    x_min (2)
    R (1)  
    u_max (1)
    u_min (1)
end

dimensions
    N = 30  
end

variables
    x[t] (2), t=1..N+1
    u[t] (1), t=0..N
end

settings
    iterations = 0
    time = 0
end

minimize 
    sum[k=0..N-1](x[k]’*Q*x[k] + u[k]’*R*u[k]) + x[N]’*P*x[N]
subject to
    x[0] = x0
    x[k+1] = A*x[k] + B*u[k], k = 0..N-1
    x_min <= x[k] <= x_max, k = 0..N-1
    u_min <= u[k] <= u_max, k= 0..N-1
    F*x[N] <= g
end
  \end{verbatim}
  \caption{Problemfilens struktur.}
  \label{fig:problemstruktur}
\end{figure}
\noindent Figur \ref{fig:problemstruktur} visar hur problemfilen är uppbyggd och hur de olika delarna ska fyllas i. Parsern är uppbyggd på ett relativt enkelt sätt vilket innebär att den förutsätter att all information om matriser så som dimensioner och matrisnamn stämmer överens mellan datafilen och problemfilen samt att matrisnamn och variabelnamn inte är valfria utan måste vara A, B, Fx, Gx, N, P, Q, R, u\_max, u\_min, k, x\_max och x\_min. Dessutom hanterar parsern endast ett givet problem vilket innebär att den inte parsar det som står under ''variables'' och ''minimize''. 
\begin{figure}[H]
  \centering
  \begin{verbatim}
A =

    0.9721    0.0155
    0.2114    0.9705
  \end{verbatim}
  \caption{Datafilens struktur.}
  \label{fig:datafilstruktur}
\end{figure}
\noindent Figur \ref{fig:datafilstruktur} visar hur en matris ska skrivas i datafilen där matrisnamnet följs av en blankrad och värdena skrivs med minst ett mellanslag mellan sig.
\newline
\newline
GUI:t består av tre huvudsakliga komponenter ett textfält som används för att skriva in eller ändra problemet som ska lösas, en knapp med texten ''C code'' och en knapp med texten ''Run code''. Den förstnämnda knappen anropar parsern med problemfilen och datafilen samt skapar filen ''solution.c''. Denna fil är uppbyggd på så sätt att det först skapas matriser som definierats i problemfilen som sedan fylls med data från datafilen. Därefeter sker anrop till funktionerna ''trans\_dyn\_cons'' och ''trans\_ineq\_cons'' som skapar matriser innehållandes ekvivalensbivilkor respektive inekvivalensbivilkor. Därefter skapas problemdatatypen ''problem'', ett anrop till lösaren samt kod för att frigöra skapade matriser.
\newline
\newline
Den andra knappen, ''Run code'', exekverar kommandot ''make \&\& ./solution'' i terminalen vilket exekverar ''solution.c'' som parsern skapar vilket i sin tur leder till ett anrop till själva lösaren och skriver till sist ut resultatet i terminalen.