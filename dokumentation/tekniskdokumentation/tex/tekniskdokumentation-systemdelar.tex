\section{Systemdelar}
QuadOpt innehåller ett flertal olika moduler som används för att lösa och tolka optimeringsproblemet, förklaringar av hur dessa är uppbyggda och fungerar kommer att ges i detta avsnitt.
(OBS kan möjligtvis vara bra att ha egna filer för dessa underrubriker om det blir mycket)

\subsection{Lösaren}
Här kan t.ex. förklaring av hur lösaren fungerar steg för steg, använd möjligtvis delar från kandidatrapporten.

\subsection{MATLAB - MEX} \label{subsec:mex}

MEX står för \textbf{M}atlab \textbf{ex}ecutable är utvecklat av MathWorks och används för att bygga MATLAB funktioner från C/C++ och Fortran funktioner. Det innehåller ett bibliotek med funktioner för att konvertera och skicka datatyper mellan MATLAB och C. För att kunna använda en C funktion i MATLAB behöver en mexfunktion användas i C filen, se figur~\ref{fig:mex2}. 

\begin{figure}[H]
\lstinputlisting[language=C]{tex/mex.c}
\caption{MEX gateway routine}
\label{fig:mex2}
\end{figure}  

Denna funktion ger tillgång till inskickade objekt från MATLAB i fältet ''prhs[]'' och utgående objekt ska läggas i fältet ''plhs[]''. Dessa objekt är av typen ''mxArray'' vilket är en datatyp som används av MATLAB. Med hjälp av olika funktioner i MEX-biblioteket kan dessa datatyper konverteras till datatyper som C kan använda och tillbaka. Några händiga funktioner följer:
\begin{itemize}
\item mxGetM(mxArray) - returnerar mxArray rader
\item mxCreateDoubleMatrix(rader, kolumner, mxREAL) - returnerar en mxArray
\end{itemize}



\subsection{Matlib}
Förklaring av viktiga saker i matrisbiblioteket, så det kan vidareutvecklas. beskrivningar av funktioner kommer finnas i doxygen dokumentet.

\lstinputlisting[language=C]{tex/matrix_struct.c}
\subsection{Byggsystem}

\subsection{Parser}
