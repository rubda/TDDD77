\section{Dokumentation}
Den tekniska dokumentationen ska användas för vidareutveckling eller modifikation av källkoden. Användarhandledning ska innehålla instruktioner för hur programmet används. Användarhandledningen ska vara lättförstålig och strukturerad så att någon med god kunskap om datorer snabbt ska kunna komma igång med programmet. Användarhandledning ska även beskriva hur man kompilerar källkoden på samtliga operativsystem som stöds.
%\subsection{Medföljande dokumentation} \label{dokumentation}
%\center
	\begin{table}[h]
	\centering
		\begin{tabularx}{\textwidth}{| l | l | X | l | l |}
			\hline
			\textbf{Dokument} & \textbf{Språk} & \textbf{Syfte} & \textbf{Målgrupp} & \textbf{Format} \\
			\hline
			Teknisk dokumentation & Svenska & Beskriv hur systemet är konstruerat & Tekniskt ansvarig & PDF \\
			\hline
			Användarhandledning & Svenska & Introduktionsbeskrivning av systemet & Användare & PDF \\
			\hline
		\end{tabularx}
	\caption{Dokumentation} \label{dokumentation:tabell}
	\end{table}
%\endcenter

\subsection{Krav på dokumentation}
\begin{LIPSkravlista}
\LIPSkrav{Original}{All dokumentation enligt Tabell \ref{dokumentation:tabell} skall levereras tre dagar före slutleveransen}{1}
\LIPSkrav{Original}{Dokumentationen skall följa LIPS-standarden}{1}
\LIPSkrav{Original}{All källkod skall vara väldokumenterad}{1}
\LIPSkrav{Original}{Dokumentation i form av kommentarer i all källkod ska finnas}{1}
\end{LIPSkravlista}
