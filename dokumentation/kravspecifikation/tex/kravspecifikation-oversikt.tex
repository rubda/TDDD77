\section{Översikt av programmet}
Programmet skall bestå av en algoritm som löser konvexa kvadratiska optimeringsproblem samt ett gränssnitt mot användaren och andra program. Gränssnittet ska vara ett terminalbaserat program som används för att läsa in filer där problemet är definierat, detta ska generera en utdatafil. Även ett gränssnitt mot Mathworks programmeringsspråk och utvecklingsmiljö Matlab ska finnas. Med hjälp av gränssnittet ska man enkelt kunna definiera problem och kalla på programmet för lösning av problemet.    

\subsection{Produktkomponenter}
Den färdiga produkten kommer innehålla följande komponenter
\begin{itemize}
\item{Källkod för programmet}
\item{Teknisk dokumentation}
\item{Användarhandledning}
\end{itemize}

\subsection{Beroenden till andra system}
QuadOpt skall kunna integreras med programmen MATLAB och ARES vilket är SAABs simuleringsmiljö för jaktflygplanet JAS 39 Gripen.

\subsection{Avgränsningar}
Skall endast lösa konvexa kvadratiska optimeringsproblem. 

\subsection{Designfilosofi}
Exaktheten och prestandan av programmet prioriteras högst, d.v.s. kunna lösa problemet korrekt och inom rimlig tidsgräns. 

\subsection{Generella krav på hela systemet}

\begin{LIPSkravlista}
\LIPSkrav{Original}{Lösa konvexa kvadratiska optimeringsproblem}{1}
\LIPSkrav{Original}{Källkod för optimeringsalgoritmen ska vara skriven i C}{1}
\LIPSkrav{Original}{Källkod för GUI ska vara skriven i Python}{1}
\LIPSkrav{Original}{Ska kunna exekveras från en körbar kompilerad fil}{1}
\LIPSkrav{Original}{Ska kunna exekveras från Matlab}{1}
\LIPSkrav{Original}{Ska kunna användas på Windows, Linux och Mac}{1}
\LIPSkrav{Original}{Alla matriser ska kunna ändras mellan iterationer}{1}
\LIPSkrav{Original}{Ska kunna ställa in hur ofta variabler uppdateras}{2}
\LIPSkrav{Original}{Programmet ska använda lösningen från den tidigare iterationen som startpunkt för lösningen i nästa iteration.}{1}

\LIPSkrav{Original}{Ett parsing-GUI ska finnas}{1}

\LIPSkrav{Original}{GUI:ts designlayout ska efterlikna CVXGEN}{1}

\LIPSkrav{Original}{En definierad standard för hur kod ska skrivas i parsingprogrammet ska finnas}{1}


\LIPSkrav{Original}{GUI:t ska generera C-filer från indata}{1}

\LIPSkrav{Original}{GUI:t ska kunna anropa lösaren}{1}





\end{LIPSkravlista}
