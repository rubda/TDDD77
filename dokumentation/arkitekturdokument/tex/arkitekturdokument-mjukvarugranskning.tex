\section{Granskningar}

\subsection{Syfte}
Denna del av dokumentet avser de granskningar som ska göras på de komponenter som tillslut kommer ge oss en slutprodukt. 

\subsection{Minimikrav}

\subsubsection{Kravgranskning}
Kravgranskning skall göras för att säkerhetsställa att kraven i "Kravspecifikation", [Ljung, 2015] är tillräckliga samt realiserbara.

\subsubsection{Artitekturgranskning}
Artitekturgranskning skall göras för att undersöka om den uppfyller det som har sagts i "Kravspecifikation", [Ljung, 2015]. Även att artitekturen håller hög kvalité samt är realiserbar inom tidsramen. 

\subsubsection{Slutgranskning}
Slutgranskning innefattar att slutprodukten granskas så att den uppfyller det som har specifierats i kravspecifikationen samt artitekturdesignen. 

\subsubsection{Dokumentgranskning}
Alla dokument som produceras av projektgruppen ska granskas för att undersöka om de håller riktlinjerna som har nämnts i detta dokument. D.v.s undersöka om de håller en god nivå. 



