\section{Aktiviteter}
Nedan följer de aktiviteter som ska utföras i projektet. 

\subsection{Utbildning}
Följande utbildning krävs för att påbörja projektet.
\begin{LIPSaktivitetslista}
	\LIPSaktivitet{1}{Konvexa kvadratiska optimeringsproblem}{}{70}
	\LIPSaktivitet{2}{Karush Kunn Tucker bivillkor}{}{42}
	\LIPSaktivitet{3}{Lagrangemultiplikatorer}{}{14}
	\LIPSaktivitet{4}{Active set-metoden}{1, 2, 3}{70}
	\LIPSaktivitet{5}{Lös enkelt testproblem för hand}{4}{7}
	\LIPSaktivitet{6}{Grundutbildning i Latex}{}{7}
	\LIPSaktivitet{7}{Grundutbildning i Git}{}{7}
	\LIPSaktivitet{8}{Grundutbildning i Trello}{}{7}
	\LIPSaktivitet{9}{Grundutbildning i Gurobi}{}{7}
	\LIPSaktivitet{10}{Grundutbildning i Matlab}{}{7}
\end{LIPSaktivitetslista}

\subsection{Filsystem}
Aktiviteter som ska utföras för hantering av in- och utdata från QuadOpt.
\begin{LIPSaktivitetslista}
	\LIPSaktivitet{11}{Definiera filformat och filstruktur}{}{10}
	\LIPSaktivitet{12}{Implementera inmatning av data till programmet}{11}{10}
	\LIPSaktivitet{13}{Implementera utmatning av data från programmet}{11}{10}
	\LIPSaktivitet{14}{Utför test av filhanteringssystemet}{12, 13}{4}
\end{LIPSaktivitetslista}

\subsection{Huvudalgoritm}
Aktiviter som ska utföras till implementation av optimeringsalgoritmen.
\begin{LIPSaktivitetslista}
	\LIPSaktivitet{15}{Implementera datastrukturer}{}{35}
	\LIPSaktivitet{16}{Implementation av matrisaritmetik (multiplikation/addition)}{15}{20}	
	\LIPSaktivitet{17}{Implementera optimeringsalgoritmen}{16}{100}
	\LIPSaktivitet{18}{Göra interna tester för att se att problemet går att lösa}{17}{20}
	\LIPSaktivitet{19}{Optimering av algoritmen}{}{140}
	
\end{LIPSaktivitetslista}
\subsection{Planering}
\begin{LIPSaktivitetslista}
\LIPSaktivitet{20}{Möte varje vecka}{}{200}
\end{LIPSaktivitetslista}

\subsection{Gränssnitt}

\begin{LIPSaktivitetslista}
	\LIPSaktivitet{21}{Definiera gränssnitt mellan modulerna}{}{30}
	\LIPSaktivitet{22}{Skapa ett användargränssnitt (Matlab/terminal)}{21}{15}
	\LIPSaktivitet{23}{Definiera och implementera layout för GUI:t}{}{70}
	\LIPSaktivitet{24}{Definiera och implementera inmatningssyntax för GUI:t}{}{140}
	\LIPSaktivitet{25}{Hantera inmatning av matriser i GUI:t}{}{105}
	\LIPSaktivitet{26}{Implementera generering av C-kod i GUI:t}{}{245}
	\LIPSaktivitet{27}{Koppla samman GUI med lösaren}{}{35}
	\LIPSaktivitet{28}{Testa gränssnitten}{22}{50}
\end{LIPSaktivitetslista}

\subsection{Byggsystem}
Ett byggsystem krävs för att smidigt kompilera C-koden till de plattformar som gruppen valt att utveckla till.
\begin{LIPSaktivitetslista}
	\LIPSaktivitet{29}{Implementering av kompilering till Linux}{}{14}
	\LIPSaktivitet{30}{Implementering av kompilering till Windows}{}{14}
	\LIPSaktivitet{31}{Implementering av kompilering till Mac}{}{7}
	\LIPSaktivitet{32}{Fixa struktur på Git}{}{1}
\end{LIPSaktivitetslista}

\subsection{Gurobi}
För att kunna se hur snabb algoritmen är krävs det ett jämförbart program. Vi har valt att jämföra Quadopt med det kommersiella programmet Gurobi.
\begin{LIPSaktivitetslista}
	\LIPSaktivitet{33}{Testa med Gurobi}{9}{15}
	\LIPSaktivitet{34}{Jämför test med egen algoritm}{28}{15}
\end{LIPSaktivitetslista}

\subsection{Dokumentation}
\begin{LIPSaktivitetslista}
	\LIPSaktivitet{35}{Testplan}{}{35}
	\LIPSaktivitet{36}{Kvalitetsplan}{}{14}
	\LIPSaktivitet{37}{Arkitektur}{}{35}
	\LIPSaktivitet{38}{Teknisk dokumentation}{Gränssnitt, huvudalgoritm och filsystem är klart}{30}
	\LIPSaktivitet{39}{Användarhandledning}{Gränssnitt och GUI är klara}{10}
\end{LIPSaktivitetslista}
