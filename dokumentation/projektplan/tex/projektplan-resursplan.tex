\section{Resursplan}

\subsection{Personer}
Projektgruppen består av medlemmar enligt tabell \ref{projektplan:resursplan-personer}
\begin{table}[h]
	\centering
		\begin{tabularx}{\textwidth}{| l | l | X | l | l | l |}
			\hline
			\textbf{Namn} & \textbf{Ansvar} & \textbf{E-post} \\
			\hline
			{Adam Sestorp} & {Teamledare} & {adase035@student.liu.se} \\\hline
			{Dennis Ljung} & {Dokumentansvarig} & {denlj069@student.liu.se} \\\hline
			{Alexander Yngve} & {Utvecklingsansvarig} & {aleyn573@student.liu.se} \\\hline
			{Martin Söderén} & {Analysansvarig} & {marso329@student.liu.se} \\\hline
			{Ruben Das} & {Kvalitetssamordnare} & {rubda680@student.liu.se} \\\hline
			{Sebastian Fast} & {Arkitekt} & {sebfa861@student.liu.se} \\\hline
			{Johan Isaksson} & {Testledare} & {johis024@student.liu.se} \\\hline

		\end{tabularx}
	\caption{Medlemmar i projektgruppen} \label{projektplan:resursplan-personer}
\end{table}

\subsection{Material}
Material nödvändigt för projektet kommer förses av kunden samt till viss del handledaren. Kunden kommer förse projektgruppen med problemexempel som kan användas vid utvecklingen samt svara på frågor rörande produkten. Handledaren bistår med handledning rörande utvecklingsprocessen och andra frågor kring projektet och dess utförande.
Gruppens medlemmar ansvarar själva för utvecklingsmiljöer och att ha tillgång till datorer att utveckla produkten på.

\subsection{Lokaler}
Projektgruppen har inga speciellt tilägnade lokaler att arbeta i utan ansvarar själva för detta.

\subsection{Ekonomi}
Projektet har en tidsgräns på 300 arbetstimmar per gruppmedlem vilket ger en total tidsgräns på 2100 arbetstimmar. Toleransen för nedlagda arbetstimmar är 10 \% vilket innebär att en gruppmedlem får lägga som minst 270 timmar och som mest 330 timmar. Dessa redovisas veckovis till handledaren i form av en tidsrapport.