\section{Milstolpar och beslutspunkter}
Milstolpar är organiserade så att grundläggande funktioner implementeras först. En milstolpe anses vara avklarad när funktionaliteten är väl testad och de underliggande funktionerna är väl dokumenterade.

\subsection{Milstolpar}
Nedan följer milstolpar uppsatta för projektet.
\begin{LIPSmilstolpar}

\LIPSmilstolpe{1}{Förstudie klar}{2015-02-16}
\LIPSmilstolpe{2}{Programmet ska ha grundläggande funktionalitet}{Iteration 1}
\LIPSmilstolpe{3}{Gränsnitt mellan systemets moduler klar}{Iteration 1}
\LIPSmilstolpe{4}{Algoritmen kan lösa ett konvext problem}{Iteration 1}
\LIPSmilstolpe{5}{Gränsnitt till Matlab klart}{Iteration 2}
\LIPSmilstolpe{5}{Parsern klar}{Iteration 2}
\LIPSmilstolpe{6}{GUI:t klart}{Iteration 3}
\LIPSmilstolpe{7}{QuadOpts prestanda är någorlunda likvärdig med prestandan hos Gurobi}{Iteration 3}
\LIPSmilstolpe{8}{Demonstration godkänd}{2015-05-27}
\end{LIPSmilstolpar}

\newpage